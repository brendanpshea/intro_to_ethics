% Preamble and document setup
\documentclass{beamer}
\usetheme{Madrid}
\usecolortheme{whale}
\setbeamertemplate{navigation symbols}{}
\setbeamertemplate{footline}[frame number]

\title{Introduction to Bioethics: The Four Principles Approach}
\author{Brendan Shea, PhD}
\date{Intro to Ethics}

\begin{document}
	
	% Slide 1
	\begin{frame}
		\titlepage
	\end{frame}
	
	% Slide 2
	\begin{frame}{Introduction to Bioethics: Navigating Moral Complexity in Healthcare}
		\begin{itemize}
			\item \textbf{Bioethics} is the systematic study of ethical issues arising from advances in biology, medicine, and healthcare.
			\item Bioethical dilemmas occur when medical capabilities create new moral questions about what we \textit{should} do versus what we \textit{can} do.
			\item Healthcare professionals face complex ethical decisions that affect patient wellbeing, autonomy, and dignity.
			\item Frameworks are needed to navigate these complexities systematically and consistently.
		\end{itemize}
		
		\begin{alertblock}{Why Bioethics Matters}
			Without ethical frameworks, healthcare decisions risk being inconsistent, arbitrary, or potentially harmful to vulnerable populations.
		\end{alertblock}
	\end{frame}
	
	% Slide 3
	\begin{frame}{Historical Development of Bioethics: From Nuremberg to Present Day}
		\begin{itemize}
			\item Modern bioethics emerged following the Nuremberg Trials (1947), which exposed unethical Nazi medical experiments.
			\item The \textbf{Nuremberg Code} established the requirement for voluntary consent in human research.
			\item The \textbf{Belmont Report} (1979) identified basic ethical principles for research with human subjects.
			\item Contemporary bioethics has expanded to address advances in genetics, reproductive technologies, and end-of-life care.
		\end{itemize}
		
		\begin{exampleblock}{Historical Example}
			The Tuskegee Syphilis Study (1932-1972) denied treatment to African American men with syphilis without their knowledge, demonstrating why informed consent and justice are essential ethical requirements.
		\end{exampleblock}
	\end{frame}
	
	% Slide 4
	\begin{frame}{The Four Principles Approach: Overview and Origins}
		\begin{itemize}
			\item The \textbf{Four Principles Approach} (also called "principlism") offers a practical framework for analyzing ethical problems in medicine.
			\item Developed by Tom Beauchamp and James Childress in their 1979 book "Principles of Biomedical Ethics."
			\item The framework provides a common, accessible language for discussing ethical issues across cultural and professional boundaries.
			\item The principles are intended to be \textit{prima facie} binding, meaning they must be fulfilled unless they conflict with another principle.
		\end{itemize}
		
		\begin{block}{The Four Core Principles}
			\begin{enumerate}
				\item \textbf{Autonomy}: Respect for individual self-determination
				\item \textbf{Beneficence}: Promoting patient welfare
				\item \textbf{Non-maleficence}: Avoiding harm
				\item \textbf{Justice}: Fair distribution of benefits and burdens
			\end{enumerate}
		\end{block}
	\end{frame}
	
	% Slides 5-8
	
	% Slide 5
	\begin{frame}{Beauchamp and Childress: Architects of Principlism}
		\begin{itemize}
			\item Tom Beauchamp and James Childress published \textbf{"Principles of Biomedical Ethics"} in 1979, now in its eighth edition.
			\item They sought a practical middle ground between abstract ethical theories and case-by-case decision making.
			\item Their framework aimed to be accessible to healthcare practitioners without requiring extensive philosophical training.
			\item The principles are derived from "considered judgments" in ordinary morality and professional ethics.
		\end{itemize}
		
		\begin{table}
			\begin{tabular}{l|l}
				\textbf{Approach} & \textbf{Key Contribution} \\
				\hline
				Utilitarian & Focus on consequences and outcomes \\
				Deontological & Focus on duties and obligations \\
				Principlism & Practical, accessible middle-ground framework \\
			\end{tabular}
		\end{table}
	\end{frame}
	
	% Slide 6
	\begin{frame}{Principle 1: Autonomy - Respecting Self-Determination}
		\begin{itemize}
			\item \textbf{Autonomy} refers to the capacity of individuals to make informed, uncoerced decisions about their own healthcare.
			\item Respecting autonomy acknowledges that patients have the right to hold views, make choices, and take actions based on personal values and beliefs.
			\item Healthcare providers have a duty to disclose information, ensure understanding, and support voluntary decision-making.
			\item The principle of autonomy provides the foundation for informed consent, truth-telling, and confidentiality.
		\end{itemize}
		
		\begin{exampleblock}{Example: Patient Refusal of Treatment}
			A Jehovah's Witness patient refuses a blood transfusion despite medical necessity. Though physicians believe this harms the patient's health interests, respecting autonomy requires accepting this informed refusal.
		\end{exampleblock}
	\end{frame}
	
	% Slide 7
	\begin{frame}{Components of Informed Consent}
		\begin{itemize}
			\item \textbf{Informed consent} is the practical application of respect for autonomy in healthcare settings.
			\item Valid consent requires: disclosure of relevant information, understanding of information, voluntariness, and competence to decide.
			\item Healthcare providers must communicate risks, benefits, alternatives, and expected outcomes in understandable language.
			\item Documentation of consent is important but secondary to the ethical requirement of a genuine informed decision process.
		\end{itemize}
		
		\begin{block}{Essential Elements for Valid Informed Consent}
			\begin{enumerate}
				\item \textbf{Disclosure}: Providing adequate, relevant information
				\item \textbf{Understanding}: Ensuring comprehension of information
				\item \textbf{Voluntariness}: Freedom from coercion or manipulation
				\item \textbf{Capacity}: Ability to make and communicate a decision
				\item \textbf{Authorization}: Clear agreement to a specific intervention
			\end{enumerate}
		\end{block}
	\end{frame}
	
	% Slide 8
	\begin{frame}{Decision-Making Capacity: Assessment and Challenges}
		\begin{itemize}
			\item \textbf{Decision-making capacity} is the ability to understand relevant information, appreciate the situation, reason about options, and communicate a choice.
			\item Capacity is decision-specific rather than global—a patient may have capacity for some decisions but not others.
			\item Mental illness, developmental disabilities, or altered consciousness may impact but do not automatically eliminate capacity.
			\item When patients lack capacity, surrogate decision-makers should use \textbf{substituted judgment} (what would the patient want?) or \textbf{best interest standard} (what promotes patient welfare?).
		\end{itemize}
		
		\begin{alertblock}{Common Assessment Errors}
			Assuming that disagreement with medical recommendations indicates lack of capacity, or that psychiatric diagnosis automatically renders patients incapable of making treatment decisions.
		\end{alertblock}
	\end{frame}
	
	% Slides 9-12
	
	% Slide 9
	\begin{frame}{Autonomy in Practice: Case Studies}
		\begin{itemize}
			\item Respecting autonomy often requires balancing competing considerations in complex clinical scenarios.
			\item Physicians must distinguish between persuasion (providing reasons) and manipulation or coercion (controlling choices).
			\item Cultural differences may influence how autonomy is perceived and valued in healthcare decision-making.
			\item Advance directives and healthcare proxies extend autonomy when decision-making capacity is lost.
		\end{itemize}
		
		\begin{exampleblock}{Case: Mr. Johnson's Advance Directive}
			Mr. Johnson has advanced dementia but previously completed an advance directive refusing artificial nutrition. His family now insists "he would want to live." The medical team must determine whether to honor his documented autonomous wishes or family requests.
		\end{exampleblock}
	\end{frame}
	
	% Slide 10
	\begin{frame}{Principle 2: Beneficence - Promoting Patient Welfare}
		\begin{itemize}
			\item \textbf{Beneficence} refers to actions that promote the well-being of others—specifically patients in healthcare contexts.
			\item This principle creates a positive obligation to act for the benefit of others, beyond merely avoiding harm.
			\item Healthcare professionals have specific beneficence obligations due to their specialized knowledge and patient vulnerability.
			\item Beneficence requires balancing benefits against risks and costs in every clinical decision.
		\end{itemize}
		
		\begin{block}{Forms of Beneficence in Healthcare}
			\begin{enumerate}
				\item \textbf{Positive beneficence}: Providing benefits (e.g., treating illness)
				\item \textbf{Utility}: Balancing benefits against costs and harms
				\item \textbf{Specific beneficence}: Special obligations to specific individuals
				\item \textbf{General beneficence}: Obligations toward all persons
			\end{enumerate}
		\end{block}
	\end{frame}
	
	% Slide 11
	\begin{frame}{Balancing Benefits and Burdens in Clinical Decision-Making}
		\begin{itemize}
			\item Clinical decisions require weighing potential benefits against potential harms, burdens, and costs.
			\item The \textbf{proportionality principle} suggests that greater risks require proportionally greater expected benefits to justify an intervention.
			\item Judgments about benefits often involve quality of life considerations, which may differ between patients and providers.
			\item Evidence-based medicine helps quantify risks and benefits but cannot determine what risks are "worth taking" for a particular patient.
		\end{itemize}
		
		\begin{table}
			\begin{tabular}{l|l}
				\textbf{Decision Factor} & \textbf{Consideration} \\
				\hline
				Probability of benefit & How likely is the intervention to help? \\
				Magnitude of benefit & How significant is the expected improvement? \\
				Duration of benefit & How long will the positive effects last? \\
				Risk/burden ratio & Do potential benefits justify the risks? \\
			\end{tabular}
		\end{table}
	\end{frame}
	
	% Slide 12
	\begin{frame}{Paternalism vs. Patient Choice: Finding Middle Ground}
		\begin{itemize}
			\item \textbf{Paternalism} involves overriding individual choices for their own good, based on beneficence without respecting autonomy.
			\item \textbf{Soft paternalism} intervenes only when actions are substantially non-voluntary (acceptable in specific circumstances).
			\item \textbf{Hard paternalism} overrides informed voluntary choices (generally rejected in contemporary bioethics).
			\item Shared decision-making offers a middle path where clinicians and patients collaborate on healthcare decisions.
		\end{itemize}
		
		\begin{alertblock}{Justified Paternalism?}
			Temporarily restraining a delirious patient to prevent self-harm may be justified (soft paternalism), while forcing a competent patient to undergo dialysis against their will is not justified (hard paternalism).
		\end{alertblock}
	\end{frame}
	% Slides 13-16
	
	% Slide 13
	\begin{frame}{Beneficence in Practice: Case Studies}
		\begin{itemize}
			\item Applying beneficence requires careful assessment of what constitutes a "benefit" from the patient's perspective.
			\item Healthcare providers must avoid imposing their own values when determining what's "best" for patients.
			\item Beneficence may conflict with resource constraints in healthcare systems with limited funding.
			\item Family interests sometimes compete with individual patient interests, creating ethical tensions.
		\end{itemize}
		
		\begin{exampleblock}{Case: Experimental Cancer Treatment}
			Dr. Rivera must decide whether to recommend an experimental treatment with a 15\% chance of extending life by 6 months but severe side effects, or supportive care focused on comfort. Beneficence requires understanding what the patient values most—longer life or quality of remaining life.
		\end{exampleblock}
	\end{frame}
	
	% Slide 14
	\begin{frame}{Principle 3: Non-maleficence - First, Do No Harm}
		\begin{itemize}
			\item \textbf{Non-maleficence} is the obligation not to inflict harm or injury intentionally (\textit{primum non nocere}: "first, do no harm").
			\item This principle is historically older than beneficence and reflected in ancient medical ethics like the Hippocratic Oath.
			\item Healthcare professionals must consider potential harms of both action and inaction in their decisions.
			\item The concept of harm encompasses physical injury, pain, disability, and psychological suffering.
		\end{itemize}
		
		\begin{block}{Key Non-maleficence Obligations}
			\begin{itemize}
				\item Do not kill
				\item Do not cause pain or suffering
				\item Do not incapacitate
				\item Do not deprive of goods necessary for life
				\item Do not impose risks of harm without adequate justification
			\end{itemize}
		\end{block}
	\end{frame}
	
	% Slide 15
	\begin{frame}{Distinguishing Harm from Side Effects and Complications}
		\begin{itemize}
			\item Medical interventions often create both benefits and harms simultaneously, requiring ethical analysis.
			\item \textbf{Intended consequences} are the direct goals of treatment, while \textbf{foreseen consequences} are expected but not the aim.
			\item The \textbf{Doctrine of Double Effect} distinguishes between intended outcomes and merely foreseen harmful side effects.
			\item Healthcare providers have a duty to minimize foreseeable harms even when they are not directly intended.
		\end{itemize}
		
		\begin{table}
			\begin{tabular}{l|l|l}
				\textbf{Type} & \textbf{Definition} & \textbf{Example} \\
				\hline
				Side effect & Expected unintended effect & Hair loss from chemotherapy \\
				Complication & Unintended negative outcome & Post-surgical infection \\
				Iatrogenic harm & Harm caused by treatment & Medication error \\
				Negligence & Failure of due care & Wrong-site surgery \\
			\end{tabular}
		\end{table}
	\end{frame}
	
	% Slide 16
	\begin{frame}{Risk Assessment and the Doctrine of Double Effect}
		\begin{itemize}
			\item The \textbf{Doctrine of Double Effect} holds that an action with both good and harmful effects may be permissible if certain conditions are met.
			\item The good effect must be intended, while the harmful effect is merely foreseen (not intended).
			\item The good effect cannot be achieved through the harmful effect—it must be independent.
			\item The proportion between good and harm must be favorable—the benefit must justify the risk.
		\end{itemize}
		
		\begin{exampleblock}{Example: Pain Management in Terminal Illness}
			Administering high-dose morphine to a terminally ill patient may relieve severe pain (intended good effect) while potentially hastening death through respiratory depression (foreseen but unintended effect). This may be ethically justified if the intention is pain relief, not hastening death.
		\end{exampleblock}
	\end{frame}
	% Slides 17-20
	
	% Slide 17
	\begin{frame}{Non-maleficence in Practice: Case Studies}
		\begin{itemize}
			\item Applying non-maleficence requires balancing risks against benefits in complex clinical situations.
			\item Withholding or withdrawing life-sustaining treatments may be consistent with non-maleficence when burdens outweigh benefits.
			\item Ethical analysis must distinguish between causing harm and allowing harm to occur when intervention would be futile.
			\item Medical futility discussions center on whether interventions that cannot achieve meaningful goals violate non-maleficence.
		\end{itemize}
		
		\begin{exampleblock}{Case: End-of-Life Decision Making}
			An 89-year-old patient with advanced dementia, kidney failure, and pneumonia develops septic shock. The medical team must decide whether CPR and ventilator support would constitute beneficial treatment or harmful intervention given the poor prognosis and suffering involved.
		\end{exampleblock}
	\end{frame}
	
	% Slide 18
	\begin{frame}{Principle 4: Justice - Fair Distribution of Resources}
		\begin{itemize}
			\item \textbf{Justice} in bioethics refers to fair, equitable, and appropriate distribution of health benefits and burdens.
			\item \textbf{Distributive justice} concerns the allocation of scarce healthcare resources across populations.
			\item Justice requires that like cases be treated alike, while relevant differences may justify differential treatment.
			\item Healthcare disparities based on race, gender, socioeconomic status, or other non-relevant factors violate justice.
		\end{itemize}
		
		\begin{block}{Types of Justice in Healthcare}
			\begin{itemize}
				\item \textbf{Distributive justice}: Fair allocation of limited resources
				\item \textbf{Rights-based justice}: Equal rights to healthcare access
				\item \textbf{Legal justice}: Following established laws and procedures
				\item \textbf{Compensatory justice}: Compensation for injuries or wrongs
			\end{itemize}
		\end{block}
	\end{frame}
	
	% Slide 19
	\begin{frame}{Distributive Justice Models in Healthcare}
		\begin{itemize}
			\item Different philosophical approaches propose competing criteria for just resource distribution.
			\item \textbf{Egalitarian} theories emphasize equal access to healthcare for all persons regardless of ability to pay.
			\item \textbf{Libertarian} theories focus on free-market distribution and individual rights to choose services.
			\item \textbf{Utilitarian} theories prioritize maximizing overall health benefits across the population.
			\item \textbf{Communitarian} theories emphasize community values and solidarity in health resource decisions.
		\end{itemize}
		
		\begin{table}
			\begin{tabular}{l|l}
				\textbf{Distribution Model} & \textbf{Key Principle} \\
				\hline
				To each equally & Every person receives identical resources \\
				To each according to need & Those with greatest medical need receive more \\
				To each according to merit & Resources based on social contribution \\
				To each according to ability to pay & Market-based distribution \\
			\end{tabular}
		\end{table}
	\end{frame}
	
	% Slide 20
	\begin{frame}{Access, Allocation, and Prioritization Frameworks}
		\begin{itemize}
			\item \textbf{Macro-allocation} concerns system-level resource distribution (e.g., healthcare budgets, insurance coverage).
			\item \textbf{Micro-allocation} involves decisions about specific patients (e.g., organ transplant waitlists, ICU beds).
			\item Transparent, consistent criteria are essential for ethical allocation decisions in scarcity situations.
			\item Procedural justice requires fair processes even when people disagree about substantive principles.
		\end{itemize}
		
		\begin{alertblock}{Ethical Challenges in Allocation}
			During the COVID-19 pandemic, hospitals faced critical decisions about ventilator allocation when demand exceeded supply. Frameworks incorporating medical benefit, age, essential worker status, and random selection were developed, raising profound questions about whose lives we prioritize and why.
		\end{alertblock}
	\end{frame}
	% Slides 21-24
	
	% Slide 21
	\begin{frame}{Justice in Practice: Case Studies}
		\begin{itemize}
			\item Applying justice principles requires moving from abstract theory to concrete healthcare decisions.
			\item Justice in individual cases must consider broader social context and structural inequalities.
			\item Resource allocation decisions involve both clinical and ethical dimensions that must be balanced.
			\item Transparency and stakeholder involvement improve the perceived fairness of difficult allocation decisions.
		\end{itemize}
		
		\begin{exampleblock}{Case: Organ Transplantation}
			Two patients need a liver transplant: a 35-year-old with alcohol-related liver disease now sober for one year, and a 65-year-old with genetic liver disease. Both have similar medical urgency. Justice requires analyzing whether age or behavioral factors should influence allocation decisions.
		\end{exampleblock}
	\end{frame}
	
	% Slide 22
	\begin{frame}{Principles in Conflict: Recognizing Ethical Dilemmas}
		\begin{itemize}
			\item Ethical dilemmas arise when two or more ethical principles conflict and cannot be simultaneously satisfied.
			\item The four principles are \textbf{prima facie} (conditional) duties that can be overridden in specific circumstances.
			\item No single principle always takes priority; relative weights depend on specific context and case features.
			\item Recognizing genuine ethical conflicts is the first step toward thoughtful resolution.
		\end{itemize}
		
		\begin{block}{Common Principle Conflicts in Healthcare}
			\begin{itemize}
				\item \textbf{Autonomy vs. Beneficence}: Patient refuses beneficial treatment
				\item \textbf{Non-maleficence vs. Justice}: Limited resources force choices about who receives care
				\item \textbf{Autonomy vs. Justice}: Individual preferences vs. fair resource distribution
				\item \textbf{Beneficence vs. Non-maleficence}: Treatments with both significant benefits and risks
			\end{itemize}
		\end{block}
	\end{frame}
	
	% Slide 23
	\begin{frame}{Balancing Competing Principles: Methodological Approaches}
		\begin{itemize}
			\item \textbf{Specification} involves refining abstract principles to apply to specific situations.
			\item \textbf{Balancing} weighs the relative importance of competing principles in a particular case.
			\item \textbf{Reflective equilibrium} seeks coherence between principles and judgments about specific cases.
			\item The goal is a justified decision, not absolute certainty about the "one right answer."
		\end{itemize}
		
		\begin{table}
			\begin{tabular}{l|l}
				\textbf{Method} & \textbf{Process} \\
				\hline
				Case comparison & Analyze similar cases and precedents \\
				Casuistry & Examine paradigm cases and draw analogies \\
				Moral deliberation & Structured discussion with all stakeholders \\
				Ethics consultation & Involve trained ethics specialists \\
			\end{tabular}
		\end{table}
	\end{frame}
	
	% Slides 25-28
	
	% Slide 25
	\begin{frame}{Cultural and Religious Perspectives on the Four Principles}
		\begin{itemize}
			\item Different cultural and religious traditions may prioritize the four principles differently.
			\item \textbf{Western bioethics} often emphasizes individual autonomy, while some \textbf{Eastern traditions} may prioritize family or community harmony.
			\item Religious perspectives may emphasize sanctity of life (non-maleficence) over quality of life considerations.
			\item Cultural humility requires recognizing that ethical frameworks themselves reflect cultural values.
		\end{itemize}
		
		\begin{exampleblock}{Example: Family Decision-Making}
			In some cultures, serious medical decisions are made collectively by families rather than by individual patients alone. Healthcare providers must navigate these dynamics while still respecting patient autonomy, requiring cultural sensitivity and flexible application of principles.
		\end{exampleblock}
	\end{frame}
	
	% Slide 26
	\begin{frame}{Common Critiques of the Four Principles Approach}
		\begin{itemize}
			\item Critics argue that the principles are too abstract to provide specific guidance in complex cases.
			\item The framework offers limited guidance on how to resolve conflicts between competing principles.
			\item The approach may overemphasize individual autonomy at the expense of relational and community values.
			\item Western philosophical origins may limit cross-cultural applicability without contextual adaptation.
		\end{itemize}
		
		\begin{alertblock}{The "Empty Ethics" Critique}
			Some critics argue that principlism is an "ethics without content" that provides a common language but lacks substantive moral guidance, allowing people with fundamentally different values to appear to agree while actually reaching decisions based on unstated values.
		\end{alertblock}
	\end{frame}
	
	% Slide 27
	\begin{frame}{Contemporary Applications: AI, Genetics, and Emerging Technologies}
		\begin{itemize}
			\item Emerging technologies present novel ethical challenges that test traditional bioethical frameworks.
			\item \textbf{Genetic editing technologies} like CRISPR raise questions about the boundaries of beneficence and non-maleficence.
			\item \textbf{Artificial intelligence} in healthcare decision-making challenges traditional notions of autonomy and informed consent.
			\item \textbf{Big data} in healthcare creates tensions between beneficence (population health) and autonomy (privacy).
		\end{itemize}
		
		\begin{table}
			\begin{tabular}{l|l}
				\textbf{Technology} & \textbf{Primary Ethical Concerns} \\
				\hline
				AI diagnosis & Transparency, bias, responsibility \\
				Genetic testing & Privacy, discrimination, incidental findings \\
				Synthetic biology & Safety, equity, human enhancement \\
				Neurotechnology & Identity, privacy, mental integrity \\
			\end{tabular}
		\end{table}
	\end{frame}
	
	% Slide 28
	\begin{frame}{Conclusion: The Ongoing Relevance of Principlist Ethics}
		\begin{itemize}
			\item The Four Principles approach provides an accessible starting point for ethical analysis in healthcare.
			\item The framework's flexibility allows application across diverse contexts while maintaining core values.
			\item Effective bioethical analysis requires complementing principlism with other perspectives like narrative ethics and cultural considerations.
			\item Bioethics is not about finding perfect solutions but about making thoughtful, justified decisions in situations of uncertainty.
		\end{itemize}
		
		\begin{block}{Key Takeaways}
			\begin{itemize}
				\item Ethical dilemmas rarely have simple solutions
				\item The principlist approach offers structure but requires interpretation
				\item Different ethical frameworks illuminate different aspects of cases
				\item Bioethics is an ongoing conversation that evolves with medicine itself
			\end{itemize}
		\end{block}
	\end{frame}
	
\end{document}