\documentclass{beamer}
\usetheme{Madrid}
\usecolortheme{whale}

\usepackage{hyperref}
\usepackage{graphicx}

\title{Virtue Ethics: Aristotle and Confucius}
\author{Brendan Shea}
\date{Spring 2025}

\begin{document}

\begin{frame}
    \titlepage
\end{frame}

\begin{frame}
    \frametitle{Introduction to Virtue Ethics}
    \begin{itemize}
        \item \textbf{Virtue ethics} is a moral philosophy that emphasizes the role of character and virtue in moral philosophy, rather than actions or their consequences. Consider how we often admire people for their integrity and wisdom rather than just their actions.
        
        \item Unlike modern ethical theories that ask "What should I do?", virtue ethics asks "What kind of person should I be?" This reflects the ancient Greek concern with character formation and excellence.
        
        \item The focus is on developing excellent character traits (\textbf{virtues}) rather than following rules or calculating consequences. This approach emphasizes moral education and character development over rigid principles.
        
        \item Virtue ethics originated in Ancient Greece but remains highly influential today, offering insights into moral psychology and the nature of human flourishing.
    \end{itemize}
\end{frame}

\begin{frame}
    \frametitle{Meet Aristotle}
    \begin{itemize}
        \item Aristotle (384-322 BCE) was a student of Plato and tutor to Alexander the Great, establishing himself as one of history's most influential philosophers through his systematic study of nearly every field of knowledge. He helped establish logic, biology, and political science as distinct disciplines.
        
        \item He founded the \textbf{Lyceum}, his famous school in Athens, where he developed his philosophical ideas through careful observation and logical analysis of the world around him.
        
        \item His ethical writings, particularly the \textit{Nicomachean Ethics}, represent the first systematic study of ethics in Western philosophy, combining practical wisdom with theoretical insight.
        
        \item Unlike Plato's abstract idealism, Aristotle grounded his philosophy in practical observation and real-world experience, making his insights particularly relevant for understanding human nature and behavior.
    \end{itemize}
\end{frame}

\begin{frame}
    \frametitle{Overview of Nicomachean Ethics}
    \begin{itemize}
        \item The \textbf{Nicomachean Ethics} is Aristotle's most complete work on ethics, named either for his father Nicomachus or his son who may have edited the work.
        
        \item The text explores the question "What is the good life for human beings?" through a systematic investigation of virtue, happiness, and human excellence.
        
        \item Aristotle argues that ethics isn't just theoretical knowledge but practical wisdom (\textbf{phronesis}) that guides us in living well, emphasizing the importance of judgment and experience in ethical decision-making.
        
        \item The work is structured as a journey from understanding the nature of happiness to exploring specific virtues, culminating in a vision of the fully realized human life.
        
        \item Aristotle sees humans as naturally social and political animals, emphasizing the importance of community and friendship in ethical development. 
    \end{itemize}
\end{frame}

\begin{frame}
    \frametitle{Eudaimonia: The Highest Good}
    \begin{itemize}
        \item \textbf{Eudaimonia} represents the highest human good, often translated as 'flourishing' or 'well-being' rather than simply 'happiness,' as it encompasses the full realization of human potential.
        
        \item Unlike temporary pleasures or external success, eudaimonia is an activity of the soul in accordance with virtue over a complete life, requiring both excellence of character and favorable circumstances.
        
        \item This highest good is desired for its own sake, while other goods (wealth, honor, pleasure) are desired for the sake of eudaimonia, making it the most final and self-sufficient end. Eudaimonia is what we want for our children, or for ourselves on our deathbed.
        
        \item Achieving eudaimonia requires developing and exercising virtues throughout one's life, suggesting that the good life is an ongoing process rather than a fixed state.
    \end{itemize}
\end{frame}

\begin{frame}
    \frametitle{The Function Argument}
    \begin{itemize}
        \item The \textbf{function argument} (\textit{ergon}) suggests that human goodness depends on fulfilling our distinctive function as rational beings capable of virtuous action.
        
        \item Just as a knife's excellence lies in cutting well, human excellence lies in reasoning well and acting according to reason, distinguishing us from other living beings. Both of these things require living in a community of other humans.
        
        \item This distinctive human function involves both theoretical contemplation and practical reasoning in everyday life, requiring the development of both intellectual and moral virtues.
        
        \item The good life, therefore, consists in excellently performing our characteristic human activities, particularly those involving reason and rational choice.
    \end{itemize}
\end{frame}

\begin{frame}
    \frametitle{The Doctrine of the Mean}
    \begin{itemize}
        \item \textbf{The mean} represents the virtuous middle ground between excess and deficiency in emotions and actions, determined by practical reason rather than mathematical calculation.
        
        \item For example, courage is the mean between cowardice (deficiency) and recklessness (excess), with the precise middle varying according to circumstances and individuals.
        
        \item Finding the mean requires \textbf{practical wisdom} to judge the right response for particular situations, making virtue more complex than simply following rules.
        
        \item The doctrine emphasizes that virtue involves not just moderation but hitting the right mark in terms of feelings, actions, and motivations in specific contexts.
    \end{itemize}
\end{frame}

\begin{frame}
    \frametitle{Moral and Intellectual Virtues}
    \begin{itemize}
        \item Aristotle distinguishes between \textbf{moral virtues} (excellence of character) and \textbf{intellectual virtues} (excellence of mind), both essential for the good life.
        
        \item Moral virtues are developed through habit and practice, requiring proper education and experience to cultivate the right emotional responses and behavioral dispositions.
        
        \item Intellectual virtues include theoretical wisdom (\textbf{sophia}) for understanding universal truths and practical wisdom (\textbf{phronesis}) for making good decisions in particular situations.
        
        \item The integration of moral and intellectual virtues is crucial for achieving eudaimonia, as good character must be guided by practical wisdom.
        
        \item For Aristotle, one cannot become a good person just by "knowing what's right." You must also "practice" this.
    \end{itemize}
\end{frame}

\begin{frame}
    \frametitle{Example Virtues I}
    \begin{itemize}
        \item \textbf{Courage} (\textit{andreia}): The mean between cowardice (fleeing from danger) and recklessness (ignoring danger), involving appropriate fear and confidence in facing genuine threats.
        
        \item \textbf{Temperance} (\textit{sophrosyne}): The mean in relation to physical pleasures, particularly those of food, drink, and sex, involving neither overindulgence nor complete abstinence. Unlike some of his later followers (some Stoics and early Christians), Aristotle does not see pleasure as inherently bad.
        
        \item \textbf{Generosity} (\textit{eleutheriotes}): The mean between wastefulness and miserliness, involving giving and receiving material goods in the right amount, to the right people, at the right time. For example, giving to charity is good, but not if it makes you unable to support yourself or your family.
    \end{itemize}
\end{frame}

\begin{frame}
    \frametitle{Example Virtues II}
    \begin{itemize}
        \item \textbf{Justice} (\textit{dikaiosyne}): Both a specific virtue concerning fair distribution and exchange, and a general virtue encompassing all other virtues in relation to others.
        
        \item \textbf{Righteous indignation}: The mean between envy and malice, involving appropriate distress at the undeserved good or bad fortune of others.
        
        \item For Aristotle, virtues like justice and righteousness mean that one \emph(should) sometimes get angry at injustice, but not too much or too little. For many of us, this is a difficult balance to strike.
        
        \item \textbf{Wittiness} (\textit{eutrapelia}): The mean between buffoonery and boorishness, involving appropriate humor and social grace in conversation and entertainment. For example, a "good" joke is one that is (genuinely!) funny but not "mean".
    \end{itemize}
\end{frame}

\begin{frame}
    \frametitle{The Role of Phronesis (Practical Wisdom)}
    \begin{itemize}
        \item \textbf{Phronesis} is the intellectual virtue that guides moral action, enabling us to determine the right course of action in particular circumstances.
        
        \item Unlike theoretical wisdom, phronesis is concerned with particulars and contingent matters, requiring both experience and good character to develop properly.
        
        \item It involves excellence in deliberation, combining universal principles with particular facts to arrive at the right action in complex situations.
        
        \item Phronesis unifies the virtues, as one cannot possess any moral virtue in its full form without the practical wisdom to guide its expression.
    \end{itemize}
\end{frame}

\begin{frame}
    \frametitle{Virtue as a Habit}
    \begin{itemize}
        \item Virtues are not innate but developed through \textbf{habituation} (\textit{ethismos}), requiring consistent practice and proper guidance from early education onward.
        
        \item We become virtuous by performing virtuous actions, just as we become skilled at sports, music, or professions through practice—the development of virtue requires both skiled instruction and dedicated time and effort.
        
        \item Habit forms character by shaping our emotional responses and practical reasoning, eventually making virtuous action feel "natural" and enjoyable. However, it doesn't start out this way.
        
        \item As children, we are dependent on our parents, teachers, or social institutions to guide our moral development. As adults, we must take responsibility for our own moral growth.
    \end{itemize}
\end{frame}

\begin{frame}
    \frametitle{The Role of Emotions}
    \begin{itemize}
        \item Unlike the Stoics, Aristotle sees \textbf{emotions} (\textit{pathe}) as integral to virtue rather than obstacles to be overcome, when properly shaped by reason.
        
        \item Virtuous action requires not just doing the right thing, but doing it with the right feelings—being genuinely pleased by noble acts and pained by base ones.
        
        \item Moral education must shape both intellectual judgment and emotional responses, creating harmony between reason and feeling in the virtuous person.
        
        \item The virtuous person's emotions are not merely controlled by reason but transformed through habituation to respond appropriately to situations.
    \end{itemize}
\end{frame}

\begin{frame}
    \frametitle{External Goods and the Good Life}
    \begin{itemize}
        \item While virtue is central to eudaimonia, Aristotle recognizes that \textbf{external goods} (\textit{ta ekta agatha}) are also necessary for the complete human life.
        
        \item These include health, wealth, friends, and good fortune—conditions that provide opportunities for virtuous action and shield against serious misfortune.
        
        \item Unlike the Cynics or Stoics, Aristotle argues that significant misfortune can impede eudaimonia, even for the virtuous person, though virtue remains valuable in itself. Extreme poverty, illness, or a bad political or social situation can make it difficult to be virtuous.
        
        \item The relationship between external goods and virtue is complex: wealth and power create opportunities for virtue but can also corrupt character if improperly used.
    \end{itemize}
\end{frame}

\begin{frame}
    \frametitle{The Virtues of a Political Animal}
    \begin{itemize}
        \item Humans as \textbf{political animals} (\textit{zoon politikon}) naturally form communities, making political virtue essential to the good life.
        
        \item Political virtue involves both leadership qualities and the capacity to be a good citizen, participating effectively in the governance of the community.
        
        \item \textbf{Justice} in its complete form can only be realized in the political community, where laws and institutions shape character and enable virtuous action.
        
        \item The best political arrangement, for Aristotle, is one that enables citizens to develop and exercise virtues in service of the common good. 
        
        \item As a practical matter, he favors a mixed constitution with elements of democracy (the people), aristocracy (The educated, wealthy elite), and monarchy (a sovereign to unite everyone).
    \end{itemize}
\end{frame}

\begin{frame}
    \frametitle{Friendship and Virtue}
    \begin{itemize}
        \item \textbf{Friendship} (\textit{philia}) in its highest form is a relationship between virtuous equals who love each other for their character rather than utility or pleasure.
        
        \item Perfect friendship serves as a mirror for self-knowledge and provides opportunities for virtuous action, making it essential for moral development.
        
        \item Through friendship, we extend our concern beyond self-interest to include others' good, expanding our capacity for virtue and contributing to political harmony.
        
        \item The three types of friendship (based on virtue, pleasure, and utility) reflect different levels of moral development and self-understanding.
    \end{itemize}
\end{frame}

\begin{frame}
    \frametitle{The Role of Moral Exemplars}
    \begin{itemize}
        \item \textbf{Moral exemplars} serve as living embodiments of virtue, providing concrete models of excellence that guide others in developing their own character.
        
        \item Unlike abstract principles or rules, exemplars demonstrate how virtues manifest in actual circumstances, showing how different virtues integrate in a complete life.
        
        \item Historical exemplars often reflect their era's particular understanding of virtue, revealing how conceptions of excellence evolve while maintaining certain core features.
        
        \item The study of exemplars involves critical engagement rather than mere imitation, as we must understand why and how their actions embodied virtue.
    \end{itemize}
\end{frame}

\begin{frame}
    \frametitle{Classical Moral Exemplars}
    \begin{itemize}
        \item \textbf{Socrates} exemplified intellectual virtue and moral courage, choosing death over compromising his principles and demonstrating that the examined life is central to virtue.
        
        \item \textbf{Marcus Aurelius} represented the Stoic ideal of combining practical governance with philosophical wisdom, showing how virtue can be exercised in positions of power.
        
        \item \textbf{Cato the Younger} was celebrated in Rome for his unwavering integrity and commitment to republican principles, even in the face of political collapse.
        
        \item These classical exemplars emphasized rational self-control, civic virtue, and the integration of philosophical understanding with practical action.
    \end{itemize}
\end{frame}

\begin{frame}
    \frametitle{Religious and Scientific Moral Exemplars}
    \begin{itemize}
        \item \textbf{Mary}, the mother of Jesus, exemplified virtues of humility, faith, and courage in Christian tradition, offering a model of moral excellence centered on spiritual rather than civic virtues.
        
        \item \textbf{Galileo} represented intellectual integrity and the courage to seek truth despite institutional opposition, exemplifying virtues crucial to scientific inquiry.
        
        \item Medieval saints like \textbf{Francis of Assisi} demonstrated how radical commitment to spiritual virtues could reshape social understanding of the good life.
        
        \item These exemplars show how different cultural contexts emphasize different aspects of virtue while maintaining the importance of integrity and courage.
    \end{itemize}
\end{frame}

\begin{frame}
    \frametitle{Modern Moral Exemplars}
    \begin{itemize}
        \item \textbf{Martin Luther King Jr.} embodied moral courage and practical wisdom in pursuing justice through non-violent resistance, demonstrating how virtue responds to systemic injustice.
        
        \item \textbf{Florence Nightingale} exemplified the integration of practical care with systemic reform, showing how virtue can transform institutional practices.
        
        \item \textbf{Nelson Mandela} demonstrated how virtues of forgiveness and reconciliation can operate at a political level while maintaining personal integrity.
        
        \item Modern exemplars often highlight how traditional virtues adapt to address contemporary challenges, particularly regarding social justice and institutional change.
    \end{itemize}
\end{frame}

\begin{frame}
    \frametitle{The Virtues of the Philosopher}
    \begin{itemize}
        \item The highest intellectual virtue is \textbf{theoretical wisdom} (\textit{sophia}), which contemplates eternal truths and represents the most divine element in human nature.
        
        \item The philosophical life achieves the most complete form of eudaimonia through the exercise of our highest rational capacities, though practical virtues remain necessary.
        
        \item The philosopher must balance contemplative excellence with practical wisdom and moral virtue, as even theoretical pursuit exists within a human community.
        
        \item Philosophical virtue involves intellectual humility, precision in thought, and the ability to grasp both universal principles and particular applications.
    \end{itemize}
\end{frame}

\begin{frame}
    \frametitle{From Greece to China}
    \begin{itemize}
        \item While Aristotle developed virtue ethics in ancient Greece, Confucius independently established a virtue-based ethical system in China during the same era (6th-5th century BCE).
        
        \item Both traditions emphasize character development over rule-following, though they differ in their understanding of the ultimate goal of moral life.
        
        \item Where Aristotle focuses on individual excellence (arete) leading to eudaimonia, Confucian thought emphasizes social harmony through moral cultivation.
        
        \item These complementary approaches offer rich insights into how virtue ethics can address both personal development and social order.
    \end{itemize}
\end{frame}

\begin{frame}
    \frametitle{Historical Context of Confucius}
    \begin{itemize}
        \item Confucius (551-479 BCE) lived during the Spring and Autumn period, when China faced political fragmentation and moral crisis similar to the instability of Greek city-states.
        \item 
        \item According to standard accounts of his life, Confucius was born into a noble family but experienced poverty and political exile, leading him to seek wisdom and moral guidance.
        
        \item Like Socrates, he was primarily a teacher who gathered disciples and emphasized moral education, though his teachings were more explicitly political.
        
        \item While Aristotle built on Plato's philosophical system, Confucius saw himself as transmitting and reinterpreting ancient wisdom from the Zhou Dynasty.
        
        \item His teachings were collected in the \textit{Analects}, which like Aristotle's works, came to us through his students' compilations.
    \end{itemize}
\end{frame}

\begin{frame}
    \frametitle{Comparing Approaches to Virtue}
    \begin{itemize}
        \item Aristotle's doctrine of the mean finds parallel in Confucian emphasis on moderation and appropriateness, though Confucius focuses more on social harmony than individual excellence.
        
        \item Where Aristotle develops a systematic theory of virtue, Confucius teaches through examples and aphorisms, emphasizing practical wisdom in concrete situations.
        
        \item Both traditions see virtue as learned through practice and habituation, but Confucianism places greater emphasis on ritual and tradition as vehicles for moral development.
        
        \item The role of practical wisdom (phronesis) in Aristotle parallels the Confucian emphasis on moral discretion and judgment in applying principles to situations.
        
        \item Both Aristotle and Confucius became central figures in their respective cultural traditions, shaping ethical thought for millenia to come. They are among the most influential philosophers in history.
    \end{itemize}
\end{frame}

\begin{frame}
    \frametitle{Core Confucian Virtues}
    \begin{itemize}
        \item \textbf{Ren} (benevolence/humaneness) serves as the supreme virtue similar to Aristotle's conception of justice, encompassing proper relation to others.
        
        \item \textbf{Yi} (righteousness/appropriateness) involves judgment similar to phronesis, determining right action in context.
        
        \item \textbf{Li} (ritual propriety) has no direct Aristotelian parallel, showing Confucianism's distinctive emphasis on cultural forms in moral development.
        
        \item \textbf{Xiao} (filial piety) reflects Confucianism's greater emphasis on family relations as fundamental to moral development.
        
        \item Additional virtues include wisdom, trustworthiness, and loyalty, each contributing to the cultivation of a well-rounded moral character.
    \end{itemize}
\end{frame}

\begin{frame}
    \frametitle{Social Structure and Moral Development}
    \begin{itemize}
        \item The \textbf{Five Relationships} provide contexts for moral development: ruler-minister, father-son, husband-wife, elder-younger, friend-friend.
        
        \item Each relationship involves reciprocal duties and virtues, with the superior party having greater responsibility for maintaining moral standards.
        
        \item Family is seen as the primary school of virtue, where one first learns empathy, respect, and proper conduct.
        
        \item These relationships are not merely social conventions but opportunities for mutual moral growth and the cultivation of virtue.
        
        \item WHile modern social structures differ greatly from those in Ancient China, the idea that social relationships are the primary context for moral development remains relevant.
    \end{itemize}
\end{frame}

\begin{frame}
    \frametitle{Education and Character Formation}
    \begin{itemize}
        \item Confucian education integrates moral, cultural, and practical learning, seeing them as inseparable aspects of character development.
        
        \item The ideal of \textbf{self-cultivation} requires both individual effort and proper guidance from teachers and texts.
        
        \item Learning involves six arts: ritual, music, archery, charioteering, calligraphy, and mathematics, each developing different aspects of character.
        
        \item The goal is to develop practical wisdom that can respond appropriately to varying situations while maintaining moral integrity.
        
        \item Again, one does not need to accept the precise syllabus of Confucian education to see the value in the idea that education should be about more than just acquiring information.
    \end{itemize}
\end{frame}

\begin{frame}
    \frametitle{The Modern Revival of Virtue Ethics}
    \begin{itemize}
        \item By the mid-20th century, moral philosophy was dominated by deontological and consequentialist approaches, with virtue ethics (of Aristotelian, Stoic, or Confucian origin) largely forgotten.
        
        \item \textbf{G.E.M. Anscombe's} "Modern Moral Philosophy" (1958) critiqued modern moral philosophy's focus on obligation and law-like ethical rules. 
        
        \item Anscombe argued for a return to Aristotelian concepts of virtue, character, and flourishing, suggesting that moral philosophy had lost its way without these foundations.
        
        \item This sparked a renewed interest in character-based approaches to ethics, leading to virtue ethics' emergence as a distinct contemporary moral theory.
    \end{itemize}
\end{frame}

\begin{frame}
    \frametitle{Key Modern Virtue Ethicists}
    \begin{itemize}
        \item \textbf{Alasdair MacIntyre's} "After Virtue" (1981) argued that the Enlightenment project of justifying morality had failed, necessitating a return to Aristotelian (and perhaps religious) virtue ethics.
        
        \item \textbf{Philippa Foot} developed a naturalistic approach to virtue ethics, arguing that moral judgments are connected to natural facts about human flourishing.
        
        \item \textbf{Rosalind Hursthouse} systematized modern virtue ethics, showing how it could provide action guidance comparable to consequentialist and deontological theories. Among other things, she has applied virtue ethics to topics such as abortion, the ethical treatment of animals, and environmental ethics.
        
        \item \textbf{Martha Nussbaum} expanded virtue ethics through engagement with literature, emotions, and political theory, while defending a modified Aristotelian approach.
    \end{itemize}

\end{frame}

\begin{frame}
    \frametitle{Care Ethics: Origins and Development}
    \begin{itemize}
        \item \textbf{Carol Gilligan's} "In a Different Voice" (1982) challenged Kohlberg's theory of moral development, identifying a distinct "ethics of care" alongside the dominant "ethics of justice."
        
        \item Initially emerging from feminist critiques of traditional moral theory, care ethics emphasizes the fundamental importance of relationships and emotional attunement.
        
        \item \textbf{Nel Noddings'} "Caring" (1984) developed a systematic philosophical approach to care ethics, emphasizing the role of empathy and relationship in moral life.
        
        \item While sometimes contrasted with virtue ethics, care ethics can be understood as identifying and developing distinctively relational virtues.
    \end{itemize}
\end{frame}

\begin{frame}
    \frametitle{Core Concepts in Care Ethics}
    \begin{itemize}
        \item \textbf{Care} is understood not just as a practice but as a moral disposition involving attentiveness, responsiveness, and responsibility to others' needs.
        
        \item \textbf{Relationality} emphasizes that humans are fundamentally interconnected, challenging the autonomous individual presumed by traditional ethical theories.
        
        \item \textbf{Particularity} focuses on the specific context and relationships involved, rather than abstract universal principles.
        
        \item \textbf{Emotional attunement} is valued as a moral capacity, contrasting with traditional philosophy's emphasis on pure reason.
    \end{itemize}
\end{frame}

\begin{frame}
    \frametitle{Care Ethics as Virtue Ethics}
    \begin{itemize}
        \item Care ethics shares with virtue ethics a focus on moral character rather than rules or consequences, emphasizing the cultivation of caring dispositions.
        
        \item The virtues of care (attentiveness, responsiveness, responsibility) can be understood as excellences of character developed through practice and habituation.
        
        \item Like virtue ethics, care ethics emphasizes practical wisdom in responding to particular situations rather than applying universal principles.
        
        \item Both approaches see moral development as occurring within relationships and communities rather than in isolation.
    \end{itemize}
\end{frame}

\begin{frame}
    \frametitle{Distinctive Features of Care Ethics}
    \begin{itemize}
        \item Care ethics places greater emphasis on \textbf{dependency and vulnerability} as fundamental features of human life, rather than focusing on the virtues of independent agents.
        
        \item While virtue ethics traditionally emphasizes individual excellence, care ethics focuses on the quality of relationships and the maintenance of networks of care.
        
        \item Care ethics critiques traditional virtue ethics' emphasis on self-sufficiency, highlighting instead the moral significance of receiving as well as giving care.
        
        \item The approach emphasizes concrete relationships over abstract character traits, though these relationships shape moral character.
    \end{itemize}
\end{frame}

\begin{frame}
    \frametitle{Applications and Implications}
    \begin{itemize}
        \item Care ethics has significantly influenced \textbf{medical ethics}, emphasizing the importance of empathy and relationship in healthcare.
        
        \item In \textbf{education}, care ethics suggests approaches focusing on emotional development and relationship-building alongside academic achievement.
        
        \item Care ethics offers distinctive approaches to \textbf{political theory}, emphasizing social policies that support networks of care and address vulnerability.
        
        \item In \textbf{environmental ethics}, care perspectives emphasize emotional connection to nature and responsibility for vulnerable ecosystems.
    \end{itemize}
\end{frame}


\begin{frame}
    \frametitle{Strengths of Virtue Ethics}
    \begin{itemize}
        \item Captures the moral significance of character and motivation, addressing aspects of moral life overlooked by purely action-focused theories.
        
        \item Provides rich resources for moral education and development, offering guidance on how to become a better person rather than just following rules.
        
        \item Better accounts for the complexity of moral life, including emotions, relationships, and practical wisdom in particular situations.
        
        \item Integrates moral theory with moral psychology, offering insights into character development and the role of habit in ethical life.
    \end{itemize}
\end{frame}

\begin{frame}
    \frametitle{Challenges for Virtue Ethics}
    \begin{itemize}
        \item The \textbf{action-guidance problem}: Critics argue virtue ethics provides insufficient guidance for specific moral decisions compared to rule-based approaches. For example, how do we know what the virtuous thing to do is in a particular situation?
        
        \item The \textbf{relativism challenge}: Defining virtues across different cultural contexts raises questions about moral objectivity and universal standards. For example, the virtues praised by Ancient Greeks, Confucians, and modern Westerners may differ.
        
        \item The \textbf{situationist critique}: Empirical psychology questions whether stable character traits exist in the way virtue ethics assumes. Some research suggests that people's behavior is more influenced by situational factors than character.
        
        \item The \textbf{systematicity problem}: Virtue ethics may lack the theoretical unity and clear decision procedures of other moral theories.
    \end{itemize}
\end{frame}

\begin{frame}
    \frametitle{Contemporary Responses}
    \begin{itemize}
        \item Modern virtue ethicists have developed sophisticated responses to the action-guidance problem, showing how virtue concepts can inform specific choices.
        
        \item Cross-cultural studies reveal common threads in virtue concepts while acknowledging legitimate cultural variation in their expression.
        
        \item New work in moral psychology suggests ways to understand character that accommodate situationist insights while preserving virtue ethics' core claims.
        
        \item Virtue ethics' apparent messiness may better reflect the actual complexity of moral life than more systematic but artificial approaches.
    \end{itemize}
\end{frame}

\begin{frame}
    \frametitle{Concluding Reflections}
    \begin{itemize}
        \item Virtue ethics offers a distinctive and valuable approach to moral philosophy, emphasizing character, development, and practical wisdom.
        
        \item Its insights complement other ethical approaches, suggesting the possibility of theoretical pluralism rather than exclusive adoption. 
        
        \item For example, in medicine, utiltiarian goals (maximizing health) can be balanced with virtue ethics (empathy, compassion) and deontological ethics (respecting patient autonomy).
        
        \item Contemporary developments in virtue ethics show its continuing relevance to modern moral challenges, from technology to climate change.
        
        \item The tradition's emphasis on character development and practical wisdom remains vital for addressing contemporary ethical challenges.
    \end{itemize}
\end{frame}

\begin{frame}
    \frametitle{Review Questions I}
    \begin{enumerate}
        \item Compare and contrast Aristotelian and Confucian approaches to virtue. How do their different cultural contexts shape their understanding of moral excellence?
        
        \item How might virtue ethics address contemporary challenges like climate change or artificial intelligence? What virtues would be particularly relevant?
        
        \item Evaluate the situationist critique of virtue ethics. How might a virtue ethicist respond while taking empirical psychology seriously?
        
        \item How does care ethics both draw on and challenge traditional virtue ethics? Consider particularly their different approaches to dependency and relationships.
    \end{enumerate}
\end{frame}

\begin{frame}
    \frametitle{Review Questions II}
    \begin{enumerate}
        \item Compare virtue ethics' approach to moral education with that of deontological or consequentialist theories. Which better prepares people for real moral challenges?
        
        \item How might virtue ethics contribute to professional ethics in fields like medicine or business? What would a virtue-based approach to professional development look like?
        
        \item Can virtue ethics adequately address systemic injustice and structural problems? How might virtuous individuals contribute to institutional change?
        
        \item Assess MacIntyre's claim that the Enlightenment project of justifying morality has failed. Does virtue ethics offer a more promising foundation for ethics?
    \end{enumerate}
\end{frame}

\end{document}