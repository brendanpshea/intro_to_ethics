\documentclass{beamer}
\usetheme{Madrid}
\usecolortheme{dolphin}
\usepackage{graphicx}
\usepackage{amsmath}
\usepackage{hyperref}
\title{Introduction to Environmental Ethics: Key Debates}
\author{Brendan Shea, PhD}
\date{Intro to Ethics}

\begin{document}
	
	\begin{frame}
		\titlepage
	\end{frame}
	
	\begin{frame}{Environmental Ethics: Expanding Our Moral Horizon}
		\begin{itemize}
			\item Environmental ethics examines the moral relationship between humans and the natural environment.
			\item Traditional ethics focused primarily on humans, but environmental ethics asks: Should our moral consideration extend beyond humans?
			\item Some argue only humans deserve direct moral consideration; others argue we should include animals, plants, or even ecosystems.
			\item This expansion debate is central to how we understand our ethical obligations to the natural world.
		\end{itemize}
		
		\begin{block}{The Expansion Question}
			Should we extend moral consideration beyond humans? If so, how far?
			\begin{itemize}
				\item To animals that can feel pain?
				\item To all living things?
				\item To ecosystems and the land?
				\item Or should we focus solely on human interests?
			\end{itemize}
		\end{block}
	\end{frame}
	
	\begin{frame}{Intrinsic vs. Instrumental Value: A Fundamental Debate}
		\begin{itemize}
			\item \textbf{Instrumental value}: Something has value only as a means to something else (e.g., a forest valued for timber).
			\item \textbf{Intrinsic value}: Something has value in and of itself, regardless of usefulness to others.
			\item The key debate: Does nature have only instrumental value for humans, or does it have intrinsic value?
			\item This distinction shapes how we approach environmental protection and policy.
		\end{itemize}
		
		\begin{example}{The Old-Growth Forest Example}
			Consider an ancient forest:
			\begin{itemize}
				\item \textbf{Instrumental view}: The forest's value lies in resources it provides: timber, recreation, ecosystem services, and aesthetic pleasure for humans.
				\item \textbf{Intrinsic value view}: The forest has value independent of human benefits—it deserves protection for its own sake, not just for what humans get from it.
			\end{itemize}
		\end{example}
	\end{frame}
	
	\begin{frame}{Anthropocentrism vs. Non-Anthropocentrism}
		\begin{itemize}
			\item \textbf{Anthropocentrism}: Humans are the central or most significant entities in the world; only human interests directly matter morally.
			\item \textbf{Non-anthropocentrism}: Moral consideration should extend to non-human entities, which have value independent of human interests.
			\item Anthropocentrists argue that only humans can be moral agents and proper subjects of moral concern.
			\item Non-anthropocentrists counter that moral consideration should depend on other criteria like sentience or being alive.
		\end{itemize}
		
		\begin{alertblock}{The Practical Question}
			\scriptsize
			Some philosophers argue this theoretical debate doesn't matter much for practice:
			\begin{itemize}
				\item Bryan Norton argues that "weak anthropocentrism" (focusing on long-term, enlightened human interests) leads to the same environmental policies as non-anthropocentrism.
				\item Others contend that truly recognizing nature's intrinsic value would fundamentally change our relationship with it.
			\end{itemize}
		\end{alertblock}
	\end{frame}
	
	\begin{frame}{Utilitarian Approaches: The Greatest Good}
		\begin{itemize}
			\item \textbf{Utilitarianism} evaluates actions based on their consequences, particularly their ability to maximize welfare or happiness.
			\item The key question for environmental utilitarianism: Whose welfare counts in our calculations?
			\item Peter Singer argues that the capacity to suffer is what matters morally, so animals deserve equal consideration of interests.
			\item This raises challenging questions about how to weigh human interests against those of other species.
		\end{itemize}
		
		\begin{block}{Singer's Argument for Animal Consideration}
			\scriptsize
			\begin{itemize}
				\item Suffering is bad regardless of who experiences it
				\item Many animals can suffer (they have interests in avoiding pain)
				\item There is no morally relevant characteristic that all humans have and all animals lack
				\item Therefore, animal suffering should count equally to comparable human suffering
				\item This doesn't mean equal treatment, but equal consideration of interests
			\end{itemize}
		\end{block}
	\end{frame}
	
	\begin{frame}{The Sentience Debate: Who Counts Morally?}
		\begin{itemize}
			\item \textbf{Sentience} refers to the capacity to experience sensations, particularly pleasure and pain.
			\item Peter Singer argues sentience is the only defensible boundary for moral consideration.
			\item Tom Regan counters that being a "subject-of-a-life" (having beliefs, desires, memory, etc.) is required for full moral status.
			\item Gary Varner suggests a three-tiered approach: persons (self-aware), sentient animals, and merely living things.
		\end{itemize}
		
		\begin{example}
			\scriptsize
			How do these approaches view different entities?
			\begin{itemize}
				\item A chimpanzee: Has interests, sentience, and some self-awareness—full consideration on most views
				\item An earthworm: Has sentience (can feel pain) but limited awareness—counts for Singer, less for Regan
				\item A plant: Alive but not sentient—no direct moral status for Singer, some status for biocentrists
				\item An ecosystem: Not a sentient individual—no direct consideration for most individualistic theories
			\end{itemize}
		\end{example}
	\end{frame}
	
	\begin{frame}{Individual vs. Ecosystem Welfare: A Utilitarian Dilemma}
		\begin{itemize}
			\item Environmental utilitarians face a dilemma: Should we prioritize individual welfare or ecosystem health?
			\item Aldo Leopold's \textbf{land ethic} suggests: "A thing is right when it tends to preserve the integrity, stability, and beauty of the biotic community."
			\item This community-focused view can conflict with the welfare of individual animals.
			\item The \textbf{predator reintroduction debate} illustrates this conflict.
		\end{itemize}
		
		\begin{alertblock}{The Wolf Reintroduction Debate}
			\begin{itemize}
				\item \textbf{Ecosystem focus}: Wolves restore natural processes and increase biodiversity
				\item \textbf{Individual animal focus}: Wolves cause suffering to prey animals
				\item \textbf{The dilemma}: Can a utilitarian coherently care about both ecosystem health and individual animal welfare?
				\item \textbf{Possible resolution}: Different levels of utilitarian analysis for different contexts
			\end{itemize}
		\end{alertblock}
	\end{frame}
	
	\begin{frame}{The Cost-Benefit Debate: Can We Put a Price on Nature?}
		\begin{itemize}
			\item Environmental policies often employ \textbf{cost-benefit analysis} to determine the most utilitarian course of action.
			\item This requires assigning monetary values to environmental goods (clean air, biodiversity, wilderness).
			\item Proponents argue this ensures efficient resource allocation and proper consideration of environmental values.
			\item Critics argue some values cannot be meaningfully monetized and the process often undervalues nature.
		\end{itemize}
		
		\begin{block}{The Monetary Valuation Problem}
			\scriptsize
			How do we value these environmental goods?
			\begin{tabular}{|p{0.3\textwidth}|p{0.6\textwidth}|}
				\hline
				\textbf{Method} & \textbf{Limitation} \\
				\hline
				Willingness to pay & Limited by ability to pay; future generations can't bid \\
				\hline
				Replacement cost & Some natural systems irreplaceable \\
				\hline
				Contingent valuation & Hypothetical questions yield unreliable answers \\
				\hline
				Travel cost & Captures only use value, not existence value \\
				\hline
			\end{tabular}
		\end{block}
	\end{frame}
	
	\begin{frame}{Kantian Ethics: Beyond Rational Beings?}
		\begin{itemize}
			\item Immanuel Kant argued we should treat rational beings as ends in themselves, never merely as means.
			\item Kant originally limited direct moral consideration to rational beings (humans), granting only indirect duties regarding animals and nature.
			\item Can Kantian ethics be extended to include non-human entities in our direct moral consideration?
			\item This question has generated significant debate among environmental philosophers.
		\end{itemize}
		
		\begin{block}{Extending Kantian Ethics to Nature}
			\scriptsize
			\begin{itemize}
				\item \textbf{Traditional Kantian view}: We have direct duties only to rational beings; we should avoid harming animals only because it might make us more likely to harm humans
				\item \textbf{Christine Korsgaard's argument}: Animals have a good of their own that matters to them; respecting rational nature requires respecting other beings' natural ends
				\item \textbf{Paul Taylor's biocentrism}: All living organisms deserve respect as "teleological centers of life" with a good of their own
			\end{itemize}
		\end{block}
	\end{frame}
	
	\begin{frame}{The Categorical Imperative: Environmental Applications}
		\begin{itemize}
			\item Kant's \textbf{Categorical Imperative}: "Act only according to that maxim by which you can at the same time will that it should become a universal law."
			\item Applied to environmental ethics: Could we universalize a maxim of environmental exploitation?
			\item If everyone polluted or exploited natural resources without restraint, would we undermine the very conditions that make human life possible?
			\item This argument suggests Kantian ethics can ground environmental protection based on duties to humanity.
		\end{itemize}
		
		\begin{example}
			\scriptsize
			Consider the maxim: "I will emit greenhouse gases without restriction whenever it benefits me."
			\begin{itemize}
				\item If everyone followed this maxim, it would lead to catastrophic climate change
				\item This would undermine the conditions for human existence and agency
				\item Therefore, this maxim fails the universalization test
				\item Kantian ethics thus prohibits unrestricted greenhouse gas emissions
			\end{itemize}
		\end{example}
	\end{frame}
	
	\begin{frame}{Animal Rights: The Moral Status Debate}
		\begin{itemize}
			\item \textbf{Rights} are strong moral protections that cannot be violated even for greater overall good.
			\item Tom Regan argues that animals with complex mental lives are "subjects-of-a-life" with inherent value and rights.
			\item Being a subject-of-a-life means having beliefs, desires, perception, memory, a sense of future, and a welfare that matters.
			\item These rights prohibit treating animals merely as resources for human use (in research, food, entertainment, etc.).
		\end{itemize}
		
		\begin{alertblock}{Regan's Rights Argument}
			\scriptsize
			\begin{itemize}
				\item Many animals (especially mammals) have complex mental lives similar to humans
				\item They are "subjects-of-a-life" with inherent value
				\item Having inherent value grants them the right not to be treated merely as resources
				\item This right cannot be overridden even for greater overall utility
				\item Therefore, practices like factory farming and animal experimentation are wrong regardless of benefits
			\end{itemize}
		\end{alertblock}
	\end{frame}
	
	\begin{frame}{Rights of Nature: Conceptual Possibilities}
		\begin{itemize}
			\item Christopher Stone famously asked: "Should Trees Have Standing?" proposing legal rights for natural entities.
			\item The debate centers on whether rights make sense for entities that cannot claim or waive them.
			\item Some argue rights require interests, which require consciousness—limiting rights to sentient beings.
			\item Others suggest we can meaningfully speak of rights for non-sentient entities that have a good of their own.
		\end{itemize}
		
		\begin{block}{Rights of Nature in Practice}
			\scriptsize
			Several jurisdictions have begun recognizing rights of nature:
			\begin{itemize}
				\item \textbf{Ecuador's Constitution} (2008): Rights for "Pachamama" (Mother Earth) to exist and maintain vital cycles
				\item \textbf{New Zealand's Te Urewera Act} (2014): Former national park recognized as legal entity with rights
				\item \textbf{Rights of the Whanganui River} (2017): River granted legal personhood, with guardians appointed to speak for its interests
				\item These provide not just symbolic recognition but legal mechanisms to protect nature's interests
			\end{itemize}
		\end{block}
	\end{frame}
	
	\begin{frame}{Virtue Ethics: Character and Nature}
		\begin{itemize}
			\item \textbf{Virtue ethics} focuses on developing excellent character traits rather than following rules or calculating consequences.
			\item Environmental virtue ethics asks: "What kind of person would relate appropriately to nature?"
			\item Key environmental virtues include respect, humility, care, gratitude, and ecological wisdom.
			\item This approach shifts focus from "What should I do?" to "How should I live?" and "What kind of person should I be?"
		\end{itemize}
		
		\begin{example}
			\scriptsize
			Environmental Virtues in Practice:
			\begin{itemize}
				\item \textbf{Simplicity}: Finding fulfillment in non-material goods rather than excessive consumption
				\item \textbf{Attentiveness}: Paying close attention to natural processes and relationships
				\item \textbf{Humility}: Recognizing human limitations and dependence on natural systems
				\item \textbf{Ecological wisdom}: Understanding how natural systems work and acting in harmony with them
				\item These virtues guide choices without requiring complex calculations or rigid rules
			\end{itemize}
		\end{example}
	\end{frame}
	
	\begin{frame}{Indigenous Environmental Ethics: Relationship-Based Approaches}
		\begin{itemize}
			\item Many Indigenous ethical traditions emphasize relationships and reciprocity rather than abstract principles.
			\item Natural entities are often understood as persons or relatives with whom humans have ongoing moral relationships.
			\item \textbf{Reciprocity} involves giving back to the land that sustains human communities.
			\item These traditions often reject sharp distinctions between humans and nature, seeing humans as embedded within the natural world.
		\end{itemize}
		
		\begin{block}{Key Features of Many Indigenous Environmental Ethics}
			\scriptsize
			\begin{itemize}
				\item \textbf{Relationality}: Moral obligations emerge from specific relationships, not abstract principles
				\item \textbf{Reciprocity}: Taking from the land creates obligations to give back
				\item \textbf{Respect}: Approaching other beings with respect regardless of their usefulness
				\item \textbf{Responsibility}: Humans have duties to maintain balance and harmony
				\item These approaches differ from Western philosophical traditions but offer valuable ethical insights
			\end{itemize}
		\end{block}
	\end{frame}
	
	\begin{frame}{Animal Ethics: The Moral Status Debate}
		\begin{itemize}
			\item The animal ethics debate centers on the moral status of animals and our obligations toward them.
			\item Peter Singer argues that sentience (ability to feel pain/pleasure) is the relevant criterion for moral consideration.
			\item Tom Regan contends that animals with complex mental lives have rights that cannot be violated even for greater good.
			\item These positions challenge traditional views that animals matter only insofar as they affect human interests.
		\end{itemize}
		
		\begin{alertblock}{The Factory Farming Debate}
			\scriptsize
			Different ethical frameworks yield different critiques of factory farming:
			\begin{itemize}
				\item \textbf{Utilitarian critique (Singer)}: The suffering of billions of animals outweighs the benefits of cheap meat
				\item \textbf{Rights-based critique (Regan)}: Factory farming violates animals' rights by treating them merely as resources
				\item \textbf{Virtue ethics critique}: Industrial animal agriculture cultivates vices of detachment and insensitivity
				\item \textbf{Religious/Indigenous critique}: Factory farming violates proper relationships with animal relatives
			\end{itemize}
		\end{alertblock}
	\end{frame}
	
	\begin{frame}{The Reform vs. Abolition Debate in Animal Ethics}
		\begin{itemize}
			\item Animal advocates debate whether to pursue incremental welfare reforms or only complete abolition of animal exploitation.
			\item \textbf{The welfare position}: We should improve conditions for animals while working toward decreased animal use.
			\item \textbf{The abolitionist position}: Welfare reforms reinforce the property status of animals and delay more fundamental change.
			\item This debate involves both ethical principles and empirical questions about effective advocacy strategies.
		\end{itemize}
		
		\begin{example}
			\scriptsize
			Consider the movement for \textbf{cage-free egg production}:
			\begin{itemize}
				\item \textbf{Welfare argument}: Cage-free systems allow hens to express natural behaviors, significantly reducing suffering
				\item \textbf{Abolitionist counter}: "Cage-free" eggs still involve exploitation, suffering, and killing of male chicks; reforms make consumers feel better about continuing harmful practices
				\item \textbf{Welfare response}: Progress occurs in stages; welfare reforms build momentum for further changes
				\item \textbf{The empirical question}: Do welfare reforms decrease animal product consumption or increase it by easing ethical concerns?
			\end{itemize}
		\end{example}
	\end{frame}
	
	\begin{frame}{Climate Ethics: Unique Ethical Challenges}
		\begin{itemize}
			\item Climate change presents unique ethical challenges due to its global scale, long time horizons, and complex causation.
			\item Unlike traditional ethical problems, climate change involves:
			\begin{itemize}
				\item Collective rather than individual actions
				\item Effects dispersed across time and space
				\item Uncertain but potentially catastrophic impacts
				\item Disproportionate impacts on vulnerable populations and future generations
			\end{itemize}
		
		\end{itemize}
		
		\begin{block}{Why Climate Change Challenges Traditional Ethics}
			\scriptsize
			\begin{itemize}
				\item \textbf{The collective action problem}: Individual contributions seem negligible, yet collectively they produce catastrophic effects
				\item \textbf{The temporal problem}: Most severe impacts will affect future generations who cannot represent their interests today
				\item \textbf{The spatial problem}: Those contributing most to the problem often face the least severe consequences
				\item \textbf{The uncertainty problem}: We must act without complete knowledge of precise impacts
			\end{itemize}
		\end{block}
	\end{frame}
	
	\begin{frame}{Climate Justice: Historical Responsibility Debate}
		\begin{itemize}
			\item Wealthy nations have contributed disproportionately to historical emissions while developing nations face significant impacts.
			\item The key question: Who should bear the costs of addressing climate change?
			\item \textbf{The historical responsibility argument}: Those who caused the problem should pay to fix it.
			\item \textbf{The ability to pay argument}: Those with the most resources should contribute the most regardless of causation.
		\end{itemize}
		
		\begin{example}
			\scriptsize
			Consider these competing perspectives:
			\begin{itemize}
				\item \textbf{Developed nations' responsibility}: The U.S. has contributed about 25\% of cumulative CO$_2$ emissions despite having only 4\% of world population
				\item \textbf{Knowledge defense}: Earlier generations didn't know their emissions would cause harm
				\item \textbf{Current emissions focus}: China now emits more annually than any other nation
				\item \textbf{Per capita perspective}: On a per-person basis, many Western nations still emit far more than developing nations
			\end{itemize}
		\end{example}
	\end{frame}
	
	\begin{frame}{Climate Justice: Equal Emissions vs. Development Rights}
		\begin{itemize}
			\item Climate justice requires determining fair allocation of the remaining "carbon budget" that avoids dangerous warming.
			\item \textbf{Equal per-capita emissions}: Each person should have an equal right to emit greenhouse gases.
			\item \textbf{Development rights}: Developing nations need carbon space to eliminate poverty.
			\item \textbf{Subsistence vs. luxury emissions}: Basic needs emissions deserve priority over consumption emissions.
		\end{itemize}
		
		\begin{alertblock}{The Development Rights Argument}
			\scriptsize
			\begin{itemize}
				\item Hundreds of millions of people lack access to electricity and modern energy services
				\item Development to meet basic needs will require some increased emissions
				\item Wealthy nations achieved development using fossil fuels without restrictions
				\item Therefore, developing nations should have priority access to the remaining carbon budget
				\item Wealthy nations should both reduce emissions more rapidly and provide technology/financial support
			\end{itemize}
		\end{alertblock}
	\end{frame}
	
	\begin{frame}{Individual vs. Collective Responsibility for Climate Change}
		\begin{itemize}
			\item Climate change raises questions about the relationship between individual actions and structural change.
			\item Some argue individual actions like reducing meat consumption or flying less are essential moral obligations.
			\item Others contend individual actions distract from necessary systemic changes in policy and infrastructure.
			\item This debate involves both empirical questions about efficacy and normative questions about moral responsibility.
		\end{itemize}
		
		\begin{block}{Arguments About Individual Climate Responsibility}
			\scriptsize
			\begin{itemize}
				\item \textbf{Individual insignificance argument}: My emissions are a tiny fraction of the problem; my reductions won't make a measurable difference
				\item \textbf{Collective responsibility counter}: Climate change is the aggregation of billions of individual choices
				\item \textbf{Symbolic action argument}: Individual actions help create social and political momentum for structural change
				\item \textbf{The integrated view}: Individual and structural changes are complementary, not competing approaches
			\end{itemize}
		\end{block}
	\end{frame}
	
	\begin{frame}{Environmental Justice: Key Dimensions}
		\begin{itemize}
			\item \textbf{Environmental justice} examines how environmental benefits and burdens are distributed across society.
			\item Environmental injustice occurs when marginalized communities face disproportionate exposure to environmental hazards.
			\item Two key dimensions of environmental justice:
			\begin{itemize}
				\item \textbf{Distributive justice}: Fair outcomes in the distribution of environmental benefits and burdens
				\item \textbf{Procedural justice}: Fair processes that allow meaningful participation by affected communities
			\end{itemize}
		\end{itemize}
		\begin{example}
			\scriptsize
			In the United States:
			\begin{itemize}
				\item Race is the strongest predictor of proximity to hazardous waste facilities
				\item Black Americans are 75\% more likely than whites to live near industrial facilities that emit toxic pollution
				\item Communities with limited political power often lack resources to oppose unwanted facilities
				\item Environmental justice movements work to address these disparities through organizing, litigation, and policy change
			\end{itemize}
		\end{example}
	\end{frame}
	
	\begin{frame}{Environmental Justice: Siting Toxic Facilities}
		\begin{itemize}
			\item Facilities handling toxic materials are disproportionately located in low-income communities and communities of color.
			\item This raises profound questions about distributive justice and fairness in environmental decision-making.
			\item \textbf{The industry argument}: Siting decisions follow economic factors like land costs, not race or class.
			\item \textbf{The environmental justice response}: Even if not intentional, the outcome produces a racial and economic pattern of environmental injustice.
		\end{itemize}
		
		\begin{alertblock}{The Warren County Case}
			\scriptsize
			This foundational environmental justice case illustrates the core issues:
			\begin{itemize}
				\item North Carolina selected Warren County—a predominantly Black, low-income community—for a PCB landfill in 1982
				\item Residents organized protests, combining civil rights and environmental concerns
				\item Studies later confirmed a pattern of hazardous waste sites disproportionately located in minority communities
				\item The case helped launch the environmental justice movement that continues today
			\end{itemize}
		\end{alertblock}
	\end{frame}
	
	\begin{frame}{Indigenous Land Rights and Conservation}
		\begin{itemize}
			\item Traditional conservation often involved removing Indigenous peoples from their lands to create "pristine" protected areas.
			\item This approach created "conservation refugees" and ignored Indigenous stewardship that maintained biodiversity for generations.
			\item The debate centers on whether human presence is compatible with conservation goals.
			\item Indigenous-led conservation represents an alternative that recognizes sovereignty and traditional ecological knowledge.
		\end{itemize}
		
		\begin{block}{Competing Conservation Models}
			\scriptsize
			\begin{itemize}
				\item \textbf{Fortress conservation}: Strictly protected areas with minimal human presence
				\item \textbf{Community-based conservation}: Local communities manage resources sustainably
				\item \textbf{Co-management}: Shared governance between state agencies and Indigenous peoples
				\item \textbf{Indigenous Protected Areas}: Conservation led by Indigenous communities based on traditional knowledge and practices
				\item Growing evidence suggests Indigenous-managed lands often maintain biodiversity as effectively as conventional protected areas
			\end{itemize}
		\end{block}
	\end{frame}
	
	\begin{frame}{The Wilderness Debate: Preservation, Conservation, and Culture}
		\begin{itemize}
			\item John Muir advocated preservation of wilderness for its spiritual and aesthetic value.
			\item Gifford Pinchot promoted conservation—the sustainable use of resources for "the greatest good for the greatest number."
			\item William Cronon challenged the wilderness concept as a cultural construct that erases Indigenous history and human-nature relationships.
		\end{itemize}
		
		\begin{example}
			\scriptsize
			The establishment of \textbf{Yosemite National Park} illustrates competing perspectives:
			\begin{itemize}
				\item \textbf{The preservationist view}: Yosemite Valley as pristine wilderness, a cathedral of nature to be protected from human influence
				\item \textbf{The historical reality}: The valley was actively managed by Indigenous peoples for thousands of years
				\item \textbf{The erasure problem}: Creating the park required removing Ahwahneechee people who had shaped the landscape
				\item \textbf{The modern challenge}: Acknowledging human history while protecting natural values
			\end{itemize}
		\end{example}
	\end{frame}
	
	\begin{frame}{Biodiversity Loss: Ethical Dimensions}
		\begin{itemize}
			\item \textbf{Biodiversity} refers to the variety of life at genetic, species, and ecosystem levels.
			\item We are currently experiencing extinction rates estimated at 100-1,000 times the natural background rate.
			\item Different ethical frameworks provide distinct reasons for protecting biodiversity.
			\item The debate involves both the value of biodiversity itself and our responsibilities toward other species.
		\end{itemize}
		
		\begin{table}
			\scriptsize
			\begin{tabular}{|p{0.3\textwidth}|p{0.6\textwidth}|}
				\hline
				\textbf{Ethical Approach} & \textbf{Argument for Biodiversity Conservation} \\
				\hline
				Utilitarian & Preserves ecosystem services and potential future benefits \\
				\hline
				Rights-based & Species have intrinsic worth and right to continue existing \\
				\hline
				Virtue Ethics & Conservation cultivates virtues of care and respect \\
				\hline
				Indigenous & Maintains relationships with other beings as relatives \\
				\hline
			\end{tabular}
		\end{table}
	\end{frame}
	
	\begin{frame}{The Biodiversity Debate: Instrumental vs. Intrinsic Value}
		\begin{itemize}
			\item The biodiversity debate centers on whether species matter only for their benefits to humans or have value in themselves.
			\item \textbf{The instrumental value argument}: We should protect biodiversity for ecosystem services, medical discoveries, and food security.
			\item \textbf{The intrinsic value argument}: Species have a right to exist regardless of their usefulness to humans.
			\item \textbf{The relational value argument}: We have responsibilities to other species based on our relationships with them.
		\end{itemize}
		
		\begin{alertblock}{The Instrumental Value Problem}
			\scriptsize
			\begin{itemize}
				\item If we value species only for their usefulness, what about seemingly "useless" species?
				\item Many imperiled species lack obvious economic or medicinal value
				\item We cannot predict which species might prove valuable in the future
				\item The instrumental approach struggles to justify protecting all biodiversity
				\item This suggests either intrinsic or relational value is needed for a comprehensive ethic
			\end{itemize}
		\end{alertblock}
	\end{frame}
	
	\begin{frame}{Case Study: Development vs. Endangered Species}
		\begin{itemize}
			\item The conflict between economic development and endangered species protection illustrates core ethical tensions.
			\item The\textbf{ Endangered Species Act} prohibits any action that would jeopardize listed species, regardless of economic cost.
			\item This approach implicitly recognizes some form of rights or intrinsic value for species.
			\item Critics argue human needs should take priority, especially in developing regions with high poverty.
		\end{itemize}
		
		\begin{example}
			\scriptsize
			\textbf{The Snail Darter Controversy.} This classic environmental ethics case (1970s) raised fundamental questions:
			\begin{itemize}
				\item Construction of Tellico Dam threatened the endangered snail darter fish
				\item The Supreme Court ruled the Endangered Species Act required halting the nearly-completed dam
				\item Congress later exempted the project, allowing the dam's completion
				\item The case forced debate: Should a small fish prevent a \$100 million project promising economic benefits?
			\end{itemize}
		\end{example}
	\end{frame}
	
	\begin{frame}{Balancing Environmental Protection and Human Needs}
		\begin{itemize}
			\item One of the most persistent challenges in environmental ethics is balancing conservation with human development needs.
			\item \textbf{The development argument}: Lifting people from poverty requires economic growth and resource use.
			\item \textbf{The environmental justice argument}: Everyone deserves both a healthy environment and their basic needs met.
			\item \textbf{The sustainability argument}: Development that undermines ecological systems ultimately harms humans too.
		\end{itemize}
		
		\begin{block}{The Amazon Rainforest Debate}
			\scriptsize
			The ongoing debate about Amazon development illustrates these tensions:
			\begin{itemize}
				\item \textbf{Development perspective}: Brazil has sovereign rights to develop its resources for economic growth
				\item \textbf{Conservation perspective}: The Amazon provides irreplaceable biodiversity and climate regulation
				\item \textbf{Indigenous perspective}: Forest peoples' rights and traditional territories must be respected
				\item \textbf{Possible resolution}: Economic incentives for conservation, sustainable development models, recognition of Indigenous land rights
			\end{itemize}
		\end{block}
	\end{frame}
	
	\begin{frame}{Toward an Integrated Environmental Ethic}
		\begin{itemize}
			\item Despite differences, most environmental ethical frameworks agree on certain fundamental principles.
			\item All recognize limits to human exploitation of nature, whether based on human interests or intrinsic value.
			\item \textbf{Sustainability} emerges as a unifying concept compatible with different ethical approaches.
			\item Environmental ethics must address both theoretical questions and practical problems.
		\end{itemize}
		
		\begin{block}{Elements of an Integrated Environmental Ethic}
			\scriptsize
			\begin{itemize}
				\item Recognition that some non-human entities deserve moral consideration
				\item Concern for future generations and intergenerational justice
				\item Commitment to environmental justice within human communities
				\item Respect for biological and cultural diversity
				\item Development of virtues that support sustainable relationships with nature
				\item Practical implementation through policies, institutions, and individual choices
			\end{itemize}
		\end{block}
	\end{frame}
	
	\begin{frame}{Discussion Questions}
		\begin{itemize}
			\item How far should we extend moral consideration beyond humans? What criteria should determine moral status?
			\item Can anthropocentric ethics adequately protect nature, or do we need to recognize intrinsic value in non-human entities?
			\item How should we balance the rights or interests of individual animals against ecosystem health?
			\item What is a fair distribution of responsibility for addressing climate change between nations and generations?
			\item How can we reconcile environmental protection with the development needs of poor communities?
			\item What environmental virtues should we cultivate, and how might they guide our choices?
		\end{itemize}

	\end{frame}
	
	\end{document}