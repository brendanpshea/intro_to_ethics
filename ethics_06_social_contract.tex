\documentclass[aspectratio=169]{beamer}

% Theme settings
\usetheme{Madrid}
\usecolortheme{default}
\setbeamertemplate{navigation symbols}{}
\setbeamertemplate{footline}[frame number]

% Include necessary packages
\usepackage{listings}
\usepackage{xcolor}

% Configure listings for ASCII art
\lstset{
  basicstyle=\ttfamily,
  columns=fixed,
  keepspaces=true,
  frame=none,
  backgroundcolor=\color{white}
}


% Packages
\usepackage{amsmath}
\usepackage{graphicx}
\usepackage{booktabs}
\usepackage{url}

% Title information
\title{Introduction to Social Contract Theory}
\subtitle{From Hobbes to Rawls}
\author{Brendan Shea, PhD}
\date{Spring 2025}

\begin{document}

% Title slide
\begin{frame}
  \titlepage
\end{frame}

% Slide 1
\begin{frame}
  \frametitle{What Is Social Contract Theory? Defining the Framework}
  
  \begin{itemize}
    \item Social contract theory examines the relationship between individuals and their government or society.
    \item \textbf{Social contract} is defined as an implicit agreement among members of society to cooperate for mutual benefit.
    \item The theory seeks to explain why rational individuals would voluntarily give up certain freedoms to secure the advantages of political order.
    \item Social contract theorists attempt to demonstrate how legitimate political authority can arise from voluntary agreement.
    \item The concept forms the philosophical foundation for many modern democratic systems of government.
  \end{itemize}
\end{frame}

% Slide 2
\begin{frame}
  \frametitle{The State of Nature: Human Existence Without Government}
  
  \begin{block}{Key Concept}
    The \textbf{state of nature} is a hypothetical condition where no government exists and individuals exist without political authority.
  \end{block}
  
  \begin{itemize}
    \item The state of nature serves as a theoretical baseline for comparing organized society.
    \item Different philosophers characterize this state in dramatically different ways:
      \begin{itemize}
        \item Some view it as peaceful and cooperative
        \item Others describe it as violent and chaotic
      \end{itemize}
    \item The perceived quality of life in this state determines what kind of social contract people would accept.
    \item It represents the counterfactual: what would happen if political society dissolved?
  \end{itemize}
\end{frame}

% Slide 3
\begin{frame}
  \frametitle{Consent and Authority: The Basis of Political Legitimacy}
  
  \begin{itemize}
    \item \textbf{Political legitimacy} refers to the moral justification for a government to exercise authority over its citizens.
    \item In social contract theory, legitimate authority derives from the consent of the governed.
    \item Types of consent that theorists debate include:
      \begin{itemize}
        \item Express consent (explicitly given)
        \item Tacit consent (implied through behavior)
        \item Hypothetical consent (what rational people would agree to)
      \end{itemize}
    \item The consent requirement addresses the fundamental question: why should any person be obligated to obey another?
    \item This principle directly challenges the divine right of kings and other non-consensual theories of government.
  \end{itemize}
\end{frame}

% Slide 4
\begin{frame}
  \frametitle{Rights and Duties: The Currency of Social Contracts}
  
  \begin{alertblock}{Important Distinction}
    In social contract theory, \textbf{rights} are claims individuals can make against others, while \textbf{duties} are obligations owed to others.
  \end{alertblock}
  
  \begin{itemize}
    \item Social contracts establish which rights are retained in society and which are surrendered.
    \item \textbf{Natural rights} are those theorized to exist prior to any government or social agreement.
    \item Political institutions are created to protect certain rights and enforce corresponding duties.
    \item The balancing of rights and duties creates the moral foundation for civil society.
    \item Different contract theorists propose different distributions of rights and responsibilities.
  \end{itemize}
\end{frame}

% Slide 5
\begin{frame}
  \frametitle{Historical Context: Why Social Contract Theory Emerged}
  
  \begin{itemize}
    \item Social contract theory gained prominence during the \textbf{Enlightenment period} (17th-18th centuries) when traditional sources of authority were being questioned.
    \item The theory emerged as a response to religious wars and political instability across Europe.
    \item It offered an alternative to the \textbf{divine right of kings} theory that justified absolute monarchy.
    \item Social contract thinking provided a rational, secular basis for political obligation.
    \item Early theorists were attempting to reconcile individual freedom with the necessity of political order.
  \end{itemize}
\end{frame}

% Slide 6
\begin{frame}
  \frametitle{Hobbes and His Times: The English Civil War}
  
  \begin{block}{Historical Context}
    \textbf{Thomas Hobbes} (1588-1679) developed his theory during the English Civil War (1642-1651), a period of violent political upheaval and regicide.
  \end{block}
  
  \begin{itemize}
    \item Hobbes witnessed firsthand the chaos that resulted from challenging established authority.
    \item His work \textit{Leviathan} (1651) was written in response to the breakdown of political order.
    \item The execution of King Charles I (1649) deeply influenced Hobbes's pessimistic view of human nature.
    \item Hobbes's primary concern was establishing a theory of stable government that could prevent civil war.
    \item His experiences led him to value security and order above all other political goods.
  \end{itemize}
\end{frame}

% Slide 7
\begin{frame}
  \frametitle{Leviathan: The Sovereign as Artificial Person}
  
  \begin{itemize}
    \item In \textit{Leviathan}, Hobbes describes the commonwealth as an "artificial person" created by agreement.
    \item The \textbf{sovereign} represents the unified will of all those who enter the social contract.
    \item Hobbes uses the biblical sea monster "Leviathan" as a metaphor for the awesome power of the state.
    \item The sovereign power is:
      \begin{itemize}
        \item Created by the people's consent
        \item Authorized to act on behalf of all citizens
        \item Necessary to maintain peace and civil order
      \end{itemize}
    \item This unified sovereign authority stands above the conflicting desires of individuals.
  \end{itemize}
\end{frame}

% Slide 8
\begin{frame}
  \frametitle{The Hobbesian State of Nature: "Nasty, Brutish, and Short"}
  
  \begin{alertblock}{Hobbes's Famous Description}
    "...the life of man [is] solitary, poor, nasty, brutish, and short."
  \end{alertblock}
  
  \begin{itemize}
    \item Hobbes presents the most pessimistic view of the \textbf{state of nature} among contract theorists.
    \item In this pre-political state, Hobbes argues that humans exist in a condition of perpetual \textbf{war of all against all}.
    \item This conflict arises from:
      \begin{itemize}
        \item Natural equality of human physical and mental capabilities
        \item Scarcity of resources in the environment
        \item Each person's right to self-preservation
      \end{itemize}
    \item Without a common power to restrain them, people live in "continual fear and danger of violent death."
    \item This bleak portrait justifies Hobbes's argument for a powerful sovereign authority.
  \end{itemize}
\end{frame}

% Slide 9
\begin{frame}
  \frametitle{Absolute Sovereignty: Why Hobbes Rejects Limited Government}
  
  \begin{itemize}
    \item Hobbes argues that the sovereign must possess \textbf{absolute power} to effectively maintain peace and security.
    \item Any limitation on sovereign authority would create a dangerous division of power.
    \item Divided power inevitably leads to:
      \begin{itemize}
        \item Conflict between competing authorities
        \item Lack of clear final judgment in disputes
        \item Return to the chaos of the state of nature
      \end{itemize}
    \item Hobbes rejects constitutional checks and balances as inherently unstable.
    \item For Hobbes, even a tyrannical government is preferable to no government at all.
  \end{itemize}
\end{frame}

% Slide 10
\begin{frame}
  \frametitle{Self-Preservation as the Foundation of Morality}
  
  \begin{block}{Hobbesian Ethics}
    For Hobbes, morality is derived from the rational desire for \textbf{self-preservation}, not from divine commands or abstract principles.
  \end{block}
  
  \begin{itemize}
    \item Hobbes's ethical theory is fundamentally concerned with survival and security.
    \item The \textbf{laws of nature} are rational precepts that promote self-preservation.
    \item The first and most important natural law is to seek peace when it can be obtained.
    \item Only when peace is secured can other goods (industry, knowledge, arts) be pursued.
    \item This foundation makes Hobbes's moral theory thoroughly materialistic and practical.
  \end{itemize}
\end{frame}

% Slide 11
\begin{frame}
  \frametitle{Critiquing Hobbes: Problems with Absolute Authority}
  
  \begin{itemize}
    \item Critics argue that Hobbes creates a cure (absolute sovereignty) that may be worse than the disease (insecurity).
    \item Hobbes's theory provides no recourse for citizens against tyrannical rule or abuse of power.
    \item The \textbf{Prisoner's Dilemma}: Without mutual assurance, rational citizens might not form a social contract.
    \item Historical evidence suggests humans in pre-political societies were not necessarily in constant warfare.
    \item Later theorists questioned whether individuals would rationally surrender all rights apart from self-preservation.
  \end{itemize}
\end{frame}

% Slide 12
\begin{frame}
  \frametitle{Locke's Historical Context: The Glorious Revolution}
  
  \begin{alertblock}{Historical Background}
    \textbf{John Locke} (1632-1704) wrote his \textit{Two Treatises of Government} (1689) in the context of the Glorious Revolution (1688), which established constitutional monarchy in England.
  \end{alertblock}
  
  \begin{itemize}
    \item Unlike Hobbes, Locke wrote in a period moving toward constitutional settlement rather than civil war.
    \item Locke's work served as a justification for limiting royal power and protecting parliamentary authority.
    \item His theories reflected the growing political influence of property-owning classes.
    \item Locke was responding to Sir Robert Filmer's defense of absolute monarchy in \textit{Patriarcha}.
    \item The relative stability of Locke's era allowed him to envision a more optimistic view of human potential for cooperation.
  \end{itemize}
\end{frame}

% Slide 13
\begin{frame}
  \frametitle{Life, Liberty, and Property: Locke's Natural Rights}
  
  \begin{itemize}
    \item Locke identifies three fundamental \textbf{natural rights} that exist prior to government and cannot be legitimately surrendered.
    \item The right to \textbf{life} is the basic entitlement to existence and self-preservation.
    \item The right to \textbf{liberty} encompasses freedom of action, thought, and conscience within the bounds of natural law.
    \item The right to \textbf{property} includes:
      \begin{itemize}
        \item The fruits of one's labor
        \item Possessions acquired through legitimate means
        \item One's own body and capacities
      \end{itemize}
    \item These rights form the moral boundary that legitimate governments must respect.
  \end{itemize}
\end{frame}

% Slide 14
\begin{frame}
  \frametitle{Locke's State of Nature: Inconvenient but Not Unbearable}
  
  \begin{block}{Locke's View of Pre-Political Society}
    Unlike Hobbes, Locke describes the state of nature as a state of \textbf{perfect freedom} and equality, governed by the law of reason.
  \end{block}
  
  \begin{itemize}
    \item In Locke's state of nature, individuals possess natural rights and moral duties toward one another.
    \item People are generally capable of following the \textbf{law of nature}, which teaches that no one should harm another in life, health, liberty, or possessions.
    \item The state is characterized by inconveniences rather than perpetual war:
      \begin{itemize}
        \item Lack of established laws
        \item Absence of impartial judges
        \item Insufficient power to enforce judgments
      \end{itemize}
    \item These inconveniences motivate rational people to form political society, not fear of violent death.
  \end{itemize}
\end{frame}

% Slide 15
\begin{frame}
  \frametitle{Limited Government and the Right to Revolution}
  
  \begin{itemize}
    \item Locke's theory explicitly rejects absolute sovereignty in favor of \textbf{limited government}.
    \item Government is entrusted with power for specific purposes: protecting natural rights and promoting the public good.
    \item Political authority is fiduciary in nature—rulers act as trustees of the people's rights.
    \item When government betrays this trust by systematically violating rights, citizens possess a \textbf{right to revolution}.
    \item This revolutionary right serves as an ultimate check on government power and a guarantee of liberty.
  \end{itemize}
\end{frame}

% Slide 16
\begin{frame}
  \frametitle{Consent of the Governed: Express vs. Tacit Consent}
  
  \begin{alertblock}{The Consent Problem}
    Locke recognizes that most citizens have never explicitly consented to their government's authority, creating a potential legitimacy problem.
  \end{alertblock}
  
  \begin{itemize}
    \item \textbf{Express consent} occurs when someone explicitly agrees to join political society.
    \item \textbf{Tacit consent} is implied through:
      \begin{itemize}
        \item Accepting the benefits of political society
        \item Using public highways
        \item Living within a territory
      \end{itemize}
    \item The tacit consent doctrine attempts to explain how obligations can exist without explicit agreement.
    \item Critics argue that tacit consent is too easily presumed and offers no genuine opportunity to refuse.
    \item This remains one of the most controversial aspects of Locke's social contract theory.
  \end{itemize}
\end{frame}

% Slide 17
\begin{frame}
  \frametitle{Locke's Legacy in Democratic Theory}
  
  \begin{itemize}
    \item Locke's ideas provided the philosophical foundation for modern \textbf{liberal democracy}.
    \item His emphasis on natural rights directly influenced the American Declaration of Independence and Constitution.
    \item Locke established the concept of government as servant, not master, of the people.
    \item His theory supports several democratic principles:
      \begin{itemize}
        \item Separation of powers
        \item Constitutional limits on government
        \item Protection of individual rights
        \item Representation of the governed
      \end{itemize}
    \item Locke's framework remains the dominant justification for democratic institutions worldwide.
  \end{itemize}
\end{frame}

% Slide 18
\begin{frame}
  \frametitle{Rawls and 20th Century Political Philosophy}
  
  \begin{block}{Reviving Contract Theory}
    \textbf{John Rawls} (1921-2002) reinvigorated social contract theory in his landmark work \textit{A Theory of Justice} (1971) after it had fallen out of favor for nearly two centuries.
  \end{block}
  
  \begin{itemize}
    \item Rawls wrote in response to the dominance of utilitarianism in Anglo-American political philosophy.
    \item He sought to develop principles of justice that protect individual rights while allowing fair economic distribution.
    \item Rawls's work emerged during the Civil Rights Movement and debates about social inequality in America.
    \item He created a sophisticated contract theory that incorporates modern economic and social theory.
    \item Rawls's approach shifted focus from historical narratives to hypothetical agreement under fair conditions.
  \end{itemize}
\end{frame}

% Slide 19
\begin{frame}
  \frametitle{A Theory of Justice: Reviving Social Contract Theory}
  
  \begin{itemize}
    \item Rawls presents justice as the first virtue of social institutions, just as truth is for systems of thought.
    \item His theory aims to provide an alternative to both libertarian minimalism and utilitarian majoritarian approaches.
    \item Rawls argues that principles of justice should be those that free and rational persons would accept in an initial position of equality.
    \item The contract in Rawls's theory is entirely \textbf{hypothetical}—it describes what principles rational agents would choose under specified conditions.
    \item This approach makes the social contract a tool for discovering principles rather than explaining historical government formation.
  \end{itemize}
\end{frame}

% Slide 20
\begin{frame}
  \frametitle{The Original Position and the Veil of Ignorance}
  
  \begin{alertblock}{Rawls's Thought Experiment}
    The \textbf{original position} is a hypothetical situation where individuals choose principles of justice without knowing their place in society.
  \end{alertblock}
  
  \begin{itemize}
    \item The \textbf{veil of ignorance} prevents choosers from knowing:
      \begin{itemize}
        \item Their social class or economic status
        \item Their natural abilities or intelligence
        \item Their conception of the good or life plans
        \item The generation to which they belong
      \end{itemize}
    \item This ignorance ensures impartiality by removing self-interest from decision-making.
    \item Behind the veil, rational choosers will select principles that benefit the least advantaged positions.
    \item This method attempts to model the moral point of view where fair terms of cooperation are determined.
    \item Rawls believes this procedure leads to principles that reflect our considered convictions about justice.
  \end{itemize}
\end{frame}

% Slide 21
\begin{frame}
  \frametitle{Rawls's Two Principles of Justice}
  
  \begin{itemize}
    \item Rawls argues that rational persons in the original position would agree on two fundamental principles:
    \item \textbf{First Principle} (Liberty Principle): Each person has an equal right to the most extensive basic liberties compatible with similar liberty for all.
    \item \textbf{Second Principle} (Difference Principle): Social and economic inequalities are to be arranged so that they are:
      \begin{itemize}
        \item To the greatest benefit of the least advantaged persons
        \item Attached to offices and positions open to all under conditions of fair equality of opportunity
      \end{itemize}
    \item The first principle has priority over the second, meaning liberty cannot be sacrificed for economic benefits.
    \item These principles establish a framework for evaluating the justice of social institutions.
  \end{itemize}
\end{frame}

% Slide 22
\begin{frame}
  \frametitle{Justice as Fairness: Ensuring Equal Basic Liberties}
  
  \begin{block}{Basic Liberties}
    The \textbf{basic liberties} Rawls identifies include political liberty, freedom of speech and assembly, liberty of conscience, freedom from arbitrary arrest, and the right to hold personal property.
  \end{block}
  
  \begin{itemize}
    \item Rawls insists that these liberties cannot be traded away for economic advantages or social benefits.
    \item Basic liberties can only be restricted for the sake of liberty itself—to secure the equal liberty of all.
    \item This principle rejects utilitarian arguments that would sacrifice individual rights for greater overall happiness.
    \item The priority of liberty establishes a form of constitutionalism where certain rights are beyond political negotiation.
    \item This aspect of Rawls's theory aligns with the liberal tradition of Locke and Kant.
  \end{itemize}
\end{frame}

% Slide 23
\begin{frame}
  \frametitle{The Difference Principle: When Inequalities Are Justified}
  
  \begin{itemize}
    \item The \textbf{difference principle} permits economic inequalities only when they benefit the least advantaged members of society.
    \item This principle represents a middle path between strict egalitarianism and unrestricted capitalism.
    \item Economic inequalities can be justified when they:
      \begin{itemize}
        \item Create incentives that increase productivity
        \item Generate innovations that improve overall welfare
        \item Lead to a system that improves the position of the worst-off
      \end{itemize}
    \item The principle requires asking: "How do social and economic arrangements affect those with the fewest advantages?"
    \item Rawls argues that rational self-interested parties would choose this principle as insurance against ending up in a disadvantaged position.
  \end{itemize}
\end{frame}

% Slide 24
\begin{frame}
  \frametitle{Social Contract Theory and Modern Democracy}
  
  \begin{alertblock}{Contemporary Relevance}
    Social contract ideas underpin modern democratic systems, providing conceptual frameworks for legitimacy, representation, and constitutional design.
  \end{alertblock}
  
  \begin{itemize}
    \item Democratic constitutions embody the social contract ideal by establishing the terms of political association.
    \item Regular elections serve as a mechanism for renewing consent and holding representatives accountable.
    \item The separation of powers reflects contractarian concerns about preventing government abuse.
    \item Bills of rights codify the natural or inalienable rights that governments must respect.
    \item Contemporary debates about voting rights, representation, and government legitimacy continue to invoke contract principles.
  \end{itemize}
\end{frame}

% Slide 25
\begin{frame}
  \frametitle{Healthcare Debates: Modern Contract Perspectives}
  
  \begin{block}{Healthcare as a Test Case}
    Healthcare policy debates illustrate fundamental differences in how modern contract theorists understand social obligations and individual rights.
  \end{block}
  
  \begin{itemize}
    \item \textbf{Neo-Hobbesian} approaches emphasize social stability and view basic healthcare as necessary for security and social peace.
    \item \textbf{Lockean libertarians} argue that healthcare is primarily a private good with minimal government involvement beyond protecting medical contracts.
    \item \textbf{Rawlsian liberals} maintain that:
      \begin{itemize}
        \item Fair equality of opportunity requires accessible healthcare
        \item The difference principle justifies redistribution to ensure universal access
        \item Healthcare represents a primary good necessary for pursuing life plans
      \end{itemize}
    \item These theoretical differences translate directly into competing policy prescriptions for insurance mandates, public funding, and delivery models.
    \item The healthcare debate reveals deeper disagreements about the proper scope of the social contract itself.
  \end{itemize}
\end{frame}

% Slide 26
\begin{frame}
  \frametitle{Immigration and Border Policy: Competing Contract Claims}
  
  \begin{itemize}
    \item Immigration policy exposes tensions between different social contract traditions regarding community boundaries.
    \item \textbf{Contemporary Hobbesians} emphasize:
      \begin{itemize}
        \item The sovereign's right to control borders as essential to security
        \item The primacy of obligations to existing citizens over non-members
        \item The importance of cultural cohesion for political stability
      \end{itemize}
    \item \textbf{Lockean approaches} stress property rights and voluntary association, sometimes supporting open labor markets but restricted access to welfare.
    \item \textbf{Rawlsian perspectives} tend to question whether birthplace is a morally relevant criterion for distributing life opportunities.
    \item These theoretical differences reflect deeper questions about whether social contracts are primarily about security, liberty, or justice.
  \end{itemize}
\end{frame}

% Slide 27
\begin{frame}
  \frametitle{Education Policy: Divergent Contract Visions}
  
  \begin{alertblock}{Education and Citizenship}
    Modern contract theorists agree that education is crucial but disagree fundamentally about its proper funding, content, and purposes.
  \end{alertblock}
  
  \begin{itemize}
    \item \textbf{Neo-Hobbesian conservatives} view education as essential for creating shared values and social cohesion.
    \item \textbf{Libertarian Lockeans} argue for educational choice, parental control, and competitive markets in education.
    \item \textbf{Rawlsian egalitarians} emphasize:
      \begin{itemize}
        \item Equal educational opportunity as a precondition for just social cooperation
        \item Resources directed toward disadvantaged students
        \item Education for democratic citizenship and mutual respect
      \end{itemize}
    \item These perspectives lead to different policies on school funding, curricular control, and educational standards.
    \item The education debate reveals how contract thinking shapes institutional design in pluralistic societies.
  \end{itemize}
\end{frame}

% Slide 28
\begin{frame}
  \frametitle{Economic Inequality: Contract Theories in Conflict}
  
  \begin{itemize}
    \item Rising economic inequality has reignited debates between different social contract traditions.
    \item \textbf{Modern Hobbesians} worry primarily about inequality that threatens social stability and political order.
    \item \textbf{Contemporary Lockeans} defend market distributions as just when they result from voluntary exchanges of legitimately acquired property.
    \item \textbf{Rawlsian theorists} advocate for:
      \begin{itemize}
        \item Regulating inequalities through the difference principle
        \item Predistribution policies that spread productive assets
        \item Political institutions that prevent economic power from undermining democratic equality
      \end{itemize}
    \item These theoretical divisions shape concrete policy disputes over taxation, welfare, inheritance, and market regulation.
    \item Each position represents a coherent application of contract principles to contemporary economic arrangements.
  \end{itemize}
\end{frame}

% Slide 29
\begin{frame}
  \frametitle{Feminist Critiques: Gender Blindness in Social Contract Theory}
  
  \begin{block}{The Sexual Contract}
    Carole Pateman argues in \textit{The Sexual Contract} (1988) that classical contract theory masks a prior \textbf{sexual contract} that subordinates women in both private and public spheres.
  \end{block}
  
  \begin{itemize}
    \item Classical theorists like Locke and Rousseau excluded women from their social contracts, assuming women belonged in the domestic sphere.
    \item For example, Locke's theory of consent never addressed how married women who couldn't own property or vote could meaningfully "consent" to government.
    \item The public/private distinction in contract theory has real-world consequences:
      \begin{itemize}
        \item Domestic violence was long considered a "private matter" outside state concern
        \item Women's unpaid care work remains unaccounted for in economic measures like GDP
        \item Contract theories traditionally ignore how family responsibilities affect political participation
      \end{itemize}
    \item These critiques help explain why political rights for women came so much later than for men in liberal democracies.
    \item Modern feminist thinkers like Susan Moller Okin ask: "What would a truly gender-neutral social contract look like?"
  \end{itemize}
\end{frame}

% Slide 30
\begin{frame}
  \frametitle{Marxist Objections: Class Conflict and Power Relations}
  
  \begin{itemize}
    \item \textbf{Marxist critics} argue that social contract theory pretends everyone enters society as equals when material conditions create vast differences in power.
    \item Consider an employment contract: While legally "voluntary," a worker facing homelessness without a job isn't making a truly free choice.
    \item Contract theory presents formal rights (like free speech) as meaningful equality while ignoring how wealth determines whose voice is actually heard.
    \item Example: Citizens theoretically have equal political rights, but campaign finance systems give wealthy donors vastly more political influence.
    \item This critique highlights why formal equality under law doesn't necessarily create substantive equality in lived experience.
  \end{itemize}
\end{frame}

% Slide 31
\begin{frame}
  \frametitle{Communitarian Challenges: The Myth of the Autonomous Individual}
  
  \begin{alertblock}{The Embedded Self}
    \textbf{Communitarians} challenge contract theorists' conception of the person as an autonomous individual whose identity exists before social relationships.
  \end{alertblock}
  
  \begin{itemize}
    \item Contract theory asks: "What would I choose if I had no prior attachments?" Communitarians ask: "Is that even possible?"
    \item Real-world example: Americans don't simply "choose" to be patriotic—they are raised with national holidays, pledges, symbols, and stories.
    \item Communitarians point to how our identities are shaped by:
      \begin{itemize}
        \item Family traditions we don't choose (like religious holidays)
        \item Cultural concepts that shape our thinking (individualism vs. collectivism)
        \item Language communities that determine how we express ourselves
      \end{itemize}
    \item This helps explain why immigrants often feel torn between cultures—our communities constitute who we are.
    \item Communitarians argue that contract theory's "unencumbered self" doesn't match our actual experience of identity formation.
  \end{itemize}
\end{frame}

% Slide 32
\begin{frame}
  \frametitle{Is Consent Even Possible? The Problem of Hypothetical Contracts}
  
  \begin{itemize}
    \item When did you explicitly consent to obey your government's laws? Most people never have—highlighting a fundamental problem with contract theory.
    \item \textbf{Actual consent} is virtually nonexistent in politics—no nation has ever asked every citizen to sign a constitution or social contract.
    \item Consider practical examples of consent problems:
      \begin{itemize}
        \item Is paying taxes truly "consenting" when the alternative is imprisonment?
        \item Does remaining in your birth country constitute meaningful agreement to its laws?
        \item Can Rawls claim that everyone would agree to his principles when real-world disagreement is widespread?
      \end{itemize}
    \item This raises a question students can relate to: If you never signed your university's honor code, are you morally bound by it simply by enrolling?
    \item These critiques show why consent theory remains both powerful and problematic—it appeals to our intuitions about fairness while struggling with practical implementation.
  \end{itemize}
\end{frame}

% Summary Slide
\begin{frame}
  \frametitle{Social Contract Theory: Key Takeaways}
  
  \begin{block}{Core Ideas Across Contract Traditions}
    Despite their differences, all social contract theories attempt to explain political legitimacy through some form of consent or agreement.
  \end{block}
  
  \begin{itemize}
    \item Social contract theory has evolved from Hobbes's security-focused absolutism to more nuanced approaches to rights and justice.
    \item The state of nature serves as a theoretical device for imagining what society without government would be like:
      \begin{itemize}
        \item Hobbes: Chaotic and violent
        \item Locke: Inconvenient but governed by natural law
        \item Rawls: Irrelevant—what matters is fair agreement conditions
      \end{itemize}
    \item Each theorist's view of human nature and ideal government reflects the political challenges of their historical context.
    \item Modern debates about healthcare, immigration, education, and economic inequality continue to be shaped by competing contract traditions.

  \end{itemize}
\end{frame}

% Discussion Questions Slide
\begin{frame}
  \frametitle{Discussion Questions}
  
  \begin{enumerate}
    \item Which state of nature seems most plausible to you—Hobbes's war of all against all or Locke's inconvenient but peaceful condition? What evidence from history or anthropology might support your view?
    
    \item Have you ever explicitly consented to be governed? If not, what makes government authority legitimate in your view?
    
    \item Would you prefer to live in a society designed according to Hobbesian, Lockean, or Rawlsian principles? What would the practical differences be in your daily life?
    
    \item How might social contract theory need to be modified to address feminist critiques about gender and family?
    
    \item Does Rawls's veil of ignorance actually produce principles everyone would agree to, or does it smuggle in liberal assumptions? Would you agree to his principles?
    
    \item Consider a contemporary issue (climate change, digital privacy, universal healthcare). How would different contract traditions approach this issue? Which approach seems most compelling to you and why?
  \end{enumerate}
\end{frame}

\end{document}