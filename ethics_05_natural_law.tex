\documentclass{beamer}
\usetheme{Madrid}
\usecolortheme{whale}
\usepackage{amsmath}
\usepackage{booktabs}
\usepackage{graphicx}
\usepackage{hyperref}

\title{Natural Law: Historical Foundations and Contemporary Relevance}
\author{Brendan Shea}
\date{Spring 2025}

\begin{document}

\begin{frame}
    \titlepage
\end{frame}

\begin{frame}{Natural Law: Ancient Roots and Contemporary Relevance}
    \begin{alertblock}{What is Natural Law?}
        \textbf{Natural law} is the philosophical view that certain moral principles are inherent in nature and can be discovered through human reason.
    \end{alertblock}
    
    \begin{itemize}
        \item Human beings across different cultures and historical periods have recognized fundamental principles of right and wrong that appear to transcend human convention.
        
        \item The earliest written legal codes, such as the \textbf{Code of Hammurabi} (circa 1750 BCE), reflected beliefs about justice being grounded in something deeper than mere human agreement.
        
        \item Contemporary debates about human rights, environmental ethics, and international law continue to draw upon natural law concepts.
    \end{itemize}
\end{frame}

\begin{frame}{Plato's Forms and Universal Justice}
    \begin{itemize}
        \item \textbf{Plato} (428-348 BCE) developed his philosophy during Athens' recovery from defeat in the Peloponnesian War, seeking stable foundations for knowledge and justice.
        
        \item The \textbf{Theory of Forms} posits that abstract concepts like justice, beauty, and goodness exist as perfect, unchanging realities that transcend the physical world.
        
        \item In \textit{The Republic}, Plato argues that justice is not merely a human convention but reflects an eternal, objective moral order that can be understood through philosophical reasoning.
        
        \item The \textbf{Form of the Good} serves as the ultimate source of all moral truth and knowledge, making objective moral knowledge possible through rational investigation.
    \end{itemize}
\end{frame}

\begin{frame}{Aristotle: Nature, Reason, and the Good Life}
    \begin{itemize}
        \item \textbf{Aristotle} (384-322 BCE), Plato's student and Alexander the Great's tutor, developed a more empirically-grounded approach to understanding nature and human flourishing.
        
        \item The concept of \textbf{teleology} suggests that everything in nature has an inherent purpose or end (\textit{telos}) that defines its proper function and excellence.
        
        \item Human beings, as rational animals, achieve their natural end through the development of moral and intellectual \textbf{virtues} that enable a life of practical wisdom.
        
        \item The \textbf{natural law tradition} draws heavily from Aristotle's insight that moral truths can be discovered by examining human nature and its proper development.
    \end{itemize}
    
    \begin{alertblock}{Key Contribution}
        Aristotle established the framework for understanding morality as grounded in human nature and discoverable through practical reason.
    \end{alertblock}
\end{frame}

\begin{frame}{Stoic Philosophy and Universal Reason}
    \begin{itemize}
        \item The \textbf{Stoic school} (founded c. 300 BCE by Zeno of Citium) emerged during the Hellenistic period when Greek city-states were losing autonomy, prompting questions about universal moral principles.
        
        \item \textbf{Logos} (divine reason) pervades the universe according to Stoic philosophy, making moral truth discoverable through rational reflection on nature.
        
        \item The Stoics developed the concept of \textbf{cosmopolitanism}, arguing that all humans share in divine reason and therefore belong to a universal moral community.
        
        \item Their emphasis on natural law as universal rational principles strongly influenced Roman jurisprudence and later Christian thought.
    \end{itemize}
    
    \begin{alertblock}{Legacy}
        Stoic ideas about universal reason and moral law transcending political boundaries remain influential in international law and human rights theory.
    \end{alertblock}
\end{frame}

\begin{frame}{Cicero: Natural Law in Roman Thought}
    \begin{itemize}
        \item \textbf{Marcus Tullius Cicero} (106-43 BCE) wrote during the crisis of the Roman Republic, seeking to defend republican values against political corruption and tyranny.
        
        \item In \textit{De Re Publica}, Cicero argues that the \textbf{true law} is right reason in agreement with nature, which is universal, unchangeable, and eternal.
        
        \item Cicero's concept of \textbf{ius gentium} (law of nations) represented an attempt to identify universal principles of justice common to all peoples.
        
        \item His writings preserved Greek philosophical ideas about natural law and transmitted them to medieval Christian thinkers.
    \end{itemize}
    
    \begin{quote}
        "True law is right reason in agreement with nature; it is of universal application, unchanging and everlasting." - Cicero, \textit{De Re Publica}
    \end{quote}
\end{frame}

\begin{frame}{Augustine: Bridging Classical and Christian Natural Law}
    \begin{itemize}
        \item \textbf{Augustine of Hippo} (354-430 CE) developed his ideas during the decline of the Western Roman Empire, wrestling with questions about divine and human law.
        
        \item He synthesized classical natural law theory with Christian theology, arguing that \textbf{eternal law} exists in the mind of God and is partially accessible to human reason.
        
        \item In \textit{City of God}, Augustine explores how natural law relates to both divine providence and human free will in shaping moral behavior.
        
        \item His concept of \textbf{just war theory} drew on natural law principles to establish moral criteria for military conflict.
    \end{itemize}
    
    \begin{alertblock}{Innovation}
        Augustine's integration of natural law with Christian theology created a framework that would dominate Western moral philosophy for nearly a millennium.
    \end{alertblock}
\end{frame}

\begin{frame}{Avicenna: Islamic Natural Law}
    \begin{itemize}
        \item \textbf{Ibn Sina} (Avicenna, 980-1037 CE) wrote during the Islamic Golden Age, synthesizing Aristotelian philosophy with Islamic theology.
        
        \item He developed a sophisticated theory of \textbf{practical intellect} (\textit{al-'aql al-'amali}) that perceives moral truths through rational reflection on nature.
        
        \item Avicenna's concept of \textbf{fitrah} suggests that human beings have an innate disposition to recognize moral truth, similar to the Western notion of natural law.
        
        \item His writings influenced both Islamic and Christian natural law traditions through Latin translations that reached medieval Europe.
    \end{itemize}
    
    \begin{alertblock}{Cross-Cultural Impact}
        Avicenna demonstrates how natural law concepts developed independently in different philosophical traditions while maintaining similar core principles.
    \end{alertblock}
\end{frame}

\begin{frame}{Thomas Aquinas: Synthesizing Faith and Reason}
    \begin{alertblock}{Historical Context}
        \textbf{Thomas Aquinas} (1225-1274) developed his natural law theory during the medieval renaissance, when newly rediscovered Aristotelian texts were transforming European thought.
    \end{alertblock}
    
    \begin{itemize}
        \item His masterwork, the \textit{Summa Theologica}, presents a comprehensive synthesis of Christian theology with Aristotelian philosophy.
        
        \item Aquinas argues that human beings can discover moral truth through:
        \begin{itemize}
            \item Natural reason examining human nature
            \item Divine revelation in scripture
            \item The teachings of the Church
        \end{itemize}
        
        \item This synthesis resolved the apparent tension between faith and reason by showing how both could lead to moral truth.
    \end{itemize}
\end{frame}

\begin{frame}{The Four Types of Law According to Aquinas}
    \begin{itemize}
        \item \textbf{Eternal Law} represents God's rational governance of all creation, the divine wisdom directing all things to their proper ends. Ex: laws of physics.
        
        \item \textbf{Natural Law} is the participation of rational creatures in eternal law through their ability to discover moral principles by reason. Ex: moral truths.
        
        \item \textbf{Human Law} consists of specific rules derived from natural law to govern particular societies. Ex: legal codes.
         
        \item \textbf{Divine Law} (revealed in scripture) guides humans to their supernatural end of eternal beatitude. Ex: the Ten Commandments.
    \end{itemize}
\end{frame}

\begin{frame}{Primary Precepts of Natural Law}
    \begin{itemize}
        \item The \textbf{first principle of practical reason} states that good is to be done and pursued, and evil avoided.
        
        \item From this principle, Aquinas derives several primary precepts that reflect basic human goods:
        \begin{itemize}
            \item Preserve human life
            \item Reproduce and educate offspring
            \item Seek truth, especially about God
            \item Live in society with others
            \item Act according to reason
        \end{itemize}
        
        \item These precepts are self-evident to all rational creatures, universally binding, and unchangeable in their fundamental aspects.
    \end{itemize}
    
    \begin{alertblock}{Key Point}
        These primary precepts form the unchanging foundation of all moral reasoning in Aquinas's system.
    \end{alertblock}
\end{frame}

\begin{frame}{Secondary Precepts and Human Law}
    \begin{itemize}
        \item \textbf{Secondary precepts} show how natural law principles are applied to concrete situations through human reasoning.
        
        \item The process of deriving these rules requires:
        \begin{itemize}
            \item Careful analysis of human nature and its needs
            \item Consideration of particular circumstances
            \item Application of practical wisdom
        \end{itemize}
        
        \item Unlike primary precepts, secondary precepts can vary across different societies and circumstances while remaining grounded in natural law.
        
        \item These precepts form the basis for human law, which must be derived from natural law to be legitimate.
    \end{itemize}
\end{frame}

\begin{frame}{Practical Reason and Synderesis}
    \begin{itemize}
        \item \textbf{Practical reason} differs from theoretical reason by focusing on what ought to be done rather than what is true.
        
        \item \textbf{Synderesis} is the natural habit of mind by which we know the basic principles of moral law without the need for investigation or proof.
        
        \item The relationship between practical reason and synderesis works in the following way:
        \begin{itemize}
            \item Synderesis provides the first principles of moral action
            \item Practical reason applies these principles to particular situations
            \item Together they enable moral decision-making in concrete circumstances
        \end{itemize}
    \end{itemize}
\end{frame}

\begin{frame}{Natural Inclinations as Moral Guides}
    \begin{alertblock}{Fundamental Concept}
        Aquinas argues that our natural inclinations point us toward genuine human goods and can serve as reliable guides for moral behavior.
    \end{alertblock}

    \begin{itemize}
        \item The \textbf{natural inclinations} include fundamental human tendencies that we share with:
        
        \item All substances:
            \begin{itemize}
                \item The inclination to preserve our own being
                \item The drive for self-preservation
            \end{itemize}
            
        \item All animals:
            \begin{itemize}
                \item The inclination to reproduce
                \item The drive to care for offspring
            \end{itemize}
            
        \item Rational creatures uniquely:
            \begin{itemize}
                \item The desire to know truth, especially about God
                \item The inclination to live in society
            \end{itemize}
    \end{itemize}
\end{frame}

\begin{frame}{The Role of Virtue in Natural Law}
    \begin{itemize}
        \item \textbf{Virtues} are stable dispositions of character that enable us to act consistently with natural law principles.
        
        \item Aquinas identifies several key virtues necessary for moral life:
            \begin{itemize}
                \item Prudence (practical wisdom in moral matters)
                \item Justice (giving each their due)
                \item Fortitude (courage in facing difficulties)
                \item Temperance (moderation in pursuing pleasures)
            \end{itemize}
        
        \item These virtues are not innate but must be developed through practice and habituation.
        
        \item The virtuous person more readily perceives moral truth and acts upon it with greater ease.
    \end{itemize}
\end{frame}

\begin{frame}{Human Nature and the Common Good}
    \begin{itemize}
        \item \textbf{Human nature} is inherently social, meaning that individual flourishing cannot be separated from the good of the community.
        
        \item The \textbf{common good} transcends the mere sum of individual goods while including them.
        
        \item Political authority derives its legitimacy from its service to the common good.
        
        \item Laws serve their proper purpose when they:
        \begin{itemize}
            \item Foster conditions for human flourishing
            \item Protect individual rights within the context of community
            \item Promote virtue among citizens
        \end{itemize}
    \end{itemize}
    
    \begin{alertblock}{Contemporary Relevance}
        This understanding of the relationship between individual and common good remains central to Catholic social teaching and some forms of political theory.
    \end{alertblock}
\end{frame}

\begin{frame}{Origins of Double Effect in Aquinas}
    \begin{alertblock}{Definition}
        The \textbf{Doctrine of Double Effect} (DDE) provides a framework for evaluating actions that have both good and bad consequences.
    \end{alertblock}

    \begin{itemize}
        \item Aquinas first articulated this principle in his discussion of self-defense in the \textit{Summa Theologica}.
        
        \item The classic example concerns whether it is permissible to kill an aggressor in self-defense:
        \begin{itemize}
            \item The good effect: preserving one's own life
            \item The bad effect: the death of the aggressor
        \end{itemize}
        
        \item Aquinas argues that such actions can be permissible if the death is not intended but is an unintended consequence of a legitimate act of self-preservation.
    \end{itemize}
\end{frame}

\begin{frame}{The Four Conditions of Double Effect}
    \begin{itemize}
        \item For an action with both good and bad effects to be morally permissible, it must satisfy four conditions:
        
        \item \textbf{Nature of the Act}:
            \begin{itemize}
                \item The action itself must not be morally wrong independent of its consequences
                \item The act must be good or morally neutral in itself
            \end{itemize}
            
        \item \textbf{Intention of the Agent}:
            \begin{itemize}
                \item The agent must intend only the good effect
                \item The bad effect must be unintended and unavoidable
            \end{itemize}
            
        \item \textbf{Distinction of Effects}:
            \begin{itemize}
                \item The bad effect must not be the means to the good effect
                \item Both effects must flow directly from the action
            \end{itemize}
            
        \item \textbf{Proportionality}:
            \begin{itemize}
                \item The good effect must outweigh the bad effect
                \item There must be a sufficiently grave reason for causing the harm
            \end{itemize}
    \end{itemize}
\end{frame}

\begin{frame}{Classic Applications: Self-Defense and Medical Ethics}
    \begin{itemize}
        \item \textbf{Self-Defense} provides the paradigmatic case for double effect reasoning:
        
        \item In medical ethics, double effect is often applied to cases such as:
        \begin{itemize}
            \item Terminal sedation of patients in severe pain
            \item Treating ectopic pregnancies
            \item High-risk pregnancies where saving the mother may result in fetal death
        \end{itemize}
        
        \item Key considerations in medical applications:
        \begin{itemize}
            \item The distinction between intended and foreseen consequences
            \item The role of professional obligations
            \item Balancing competing goods
        \end{itemize}
    \end{itemize}
    
    \begin{alertblock}{Medical Context}
        The principle helps guide difficult clinical decisions where treatments may have both beneficial and harmful effects.
    \end{alertblock}
\end{frame}

\begin{frame}{Contemporary Debates in Double Effect}
    \begin{itemize}
        \item Contemporary philosophers have raised several important challenges to double effect reasoning:
        
        \item \textbf{The Intention Problem}:
            \begin{itemize}
                \item How can we reliably distinguish between intended and merely foreseen consequences?
                \item Does the distinction matter morally?
            \end{itemize}
        
        \item \textbf{The Closeness Problem}:
            \begin{itemize}
                \item When are bad effects too close to good effects to be permissible?
                \item Can we meaningfully separate means from side effects?
            \end{itemize}
        
        \item New applications in modern contexts:
            \begin{itemize}
                \item Military targeting and civilian casualties
                \item Research ethics and risk assessment
                \item Environmental policy decisions
            \end{itemize}
    \end{itemize}
\end{frame}

\begin{frame}{Natural Law Foundations of Just War Theory}
    \begin{alertblock}{Origins}
        \textbf{Just War Theory} emerged from natural law thinking about the moral use of force, developed through Augustine, Aquinas, and later theorists.
    \end{alertblock}

    \begin{itemize}
        \item The theory addresses two fundamental questions derived from natural law principles:
        \begin{itemize}
            \item When is it morally permissible to go to war? (\textit{jus ad bellum})
            \item What actions are morally permissible in warfare? (\textit{jus in bello})
        \end{itemize}
        
        \item Natural law provides the theoretical foundation by establishing:
            \begin{itemize}
                \item The right of political communities to self-defense
                \item The obligation to protect the innocent
                \item The requirement of proportionality in the use of force
            \end{itemize}
    \end{itemize}
\end{frame}

\begin{frame}{Jus ad Bellum: Justice in Going to War}
    \begin{itemize}
        \item \textbf{Just Cause} must exist for war to be morally permissible:
        \begin{itemize}
            \item Self-defense against armed attack
            \item Defense of others against grave injustice
            \item Recovery of what has been wrongfully taken
        \end{itemize}
        
        \item \textbf{Right Intention} requires that the war be fought for the just cause and not other motives.
        
        \item \textbf{Proper Authority} means that only legitimate political authorities can declare war.
        
        \item \textbf{Proportionality} requires that the expected benefits outweigh the anticipated harms.
        
        \item \textbf{Last Resort} means that all peaceful alternatives must be exhausted first.
    \end{itemize}
\end{frame}

\begin{frame}{Jus in Bello: Justice in Conducting War}
    \begin{itemize}
        \item \textbf{Discrimination} requires distinguishing between combatants and non-combatants.
        
        \item Military actions must respect the principle of \textbf{non-combatant immunity}:
        \begin{itemize}
            \item Civilians cannot be directly targeted
            \item Civilian casualties must be unintended
            \item Reasonable precautions must be taken to minimize civilian harm
        \end{itemize}
        
        \item \textbf{Proportionality in conduct} requires that:
        \begin{itemize}
            \item Military actions must not cause excessive harm
            \item The force used must be proportional to military objectives
            \item Unnecessary suffering must be avoided
        \end{itemize}
    \end{itemize}
    
    \begin{alertblock}{Modern Application}
        These principles form the basis of international humanitarian law and the laws of armed conflict.
    \end{alertblock}
\end{frame}

\begin{frame}{Just War Theory in the Modern World}
    \begin{itemize}
        \item Contemporary challenges to traditional just war theory include:
        
        \item New forms of warfare:
        \begin{itemize}
            \item Cyber warfare and its relationship to traditional concepts of force
            \item Autonomous weapons systems and moral responsibility
            \item Non-state actors and asymmetric warfare
        \end{itemize}
        
        \item Modern just war theorists must address:
            \begin{itemize}
                \item Humanitarian intervention
                \item Preventive war
                \item Global terrorism
            \end{itemize}
        
        \item The enduring relevance of natural law principles in addressing:
            \begin{itemize}
                \item The moral status of civilian immunity
                \item Proportionality in new contexts
                \item The relationship between justice and peace
            \end{itemize}
    \end{itemize}
\end{frame}

\begin{frame}{New Natural Law Theory: Foundations}
    \begin{alertblock}{Key Innovation}
        \textbf{New Natural Law Theory} represents a significant attempt to ground natural law in practical reason without depending on metaphysical or theological premises.
    \end{alertblock}

    \begin{itemize}
        \item Developed by Germain Grisez, John Finnis, and Joseph Boyle in response to:
        \begin{itemize}
            \item Challenges from modern analytical philosophy
            \item The need for secular justification of moral claims
            \item Questions about the relationship between fact and value
        \end{itemize}
        
        \item The theory argues that basic goods are:
        \begin{itemize}
            \item Self-evident to practical reason
            \item Irreducible to one another
            \item Equally fundamental
        \end{itemize}
    \end{itemize}
\end{frame}

\begin{frame}{New Natural Law Theory: Basic Human Goods}
    \begin{itemize}
        \item \textbf{Basic human goods} are fundamental aspects of human flourishing that:
        
        \item Provide reasons for action:
        \begin{itemize}
            \item Life and health
            \item Knowledge and aesthetic experience
            \item Excellence in work and play
            \item Harmony between persons
            \item Harmony between emotions and judgment
            \item Practical reasonableness
            \item Religion and spirituality
        \end{itemize}
        
        \item These goods are:
            \begin{itemize}
                \item Incommensurable (cannot be ranked)
                \item Pre-moral (provide basis for moral reasoning)
                \item Universal (apply to all humans)
            \end{itemize}
    \end{itemize}
\end{frame}

\begin{frame}{Natural Law and Legal Theory}
    \begin{itemize}
        \item \textbf{Legal naturalism}, developed by scholars like Michael Moore, argues that:
        
        \item Law necessarily connects to moral truth through:
        \begin{itemize}
            \item The nature of legal interpretation
            \item The role of practical reason in law
            \item The function of legal systems
        \end{itemize}
        
        \item Key claims about legal meaning:
            \begin{itemize}
                \item Legal terms refer to real moral properties
                \item Judges discover rather than create legal content
                \item Moral reality constrains legal interpretation
            \end{itemize}
    \end{itemize}
    
    \begin{alertblock}{Central Insight}
        Legal meaning is not purely conventional but connects to objective moral reality that judges must work to discover.
    \end{alertblock}
\end{frame}

\begin{frame}{Natural Law in Contemporary Jurisprudence}
    \begin{itemize}
        \item Contemporary natural law approaches to legal interpretation emphasize:
        
        \item The role of practical reason in law:
        \begin{itemize}
            \item Understanding law as a rational enterprise
            \item Identifying the point or purpose of legal rules
            \item Developing coherent interpretive frameworks
        \end{itemize}
        
        \item Applications to concrete legal issues:
            \begin{itemize}
                \item Constitutional interpretation
                \item Human rights law
                \item Criminal law theory
            \end{itemize}
        
        \item Challenges to legal positivism:
            \begin{itemize}
                \item The separation of law and morality
                \item The nature of legal authority
                \item The grounds of legal obligation
            \end{itemize}
    \end{itemize}
\end{frame}

\begin{frame}{The Naturalistic Fallacy Challenge}
    \begin{alertblock}{Key Challenge}
        The \textbf{naturalistic fallacy}, identified by G.E. Moore, questions whether we can derive moral 'ought' statements from factual 'is' statements about human nature.
    \end{alertblock}

    \begin{itemize}
        \item Natural law theorists have responded to this challenge by:
        \begin{itemize}
            \item Arguing that practical reason grasps basic goods directly
            \item Showing how nature can be normative without committing the fallacy
            \item Demonstrating how facts about human nature can inform moral reasoning
        \end{itemize}
        
        \item This debate raises fundamental questions about:
        \begin{itemize}
            \item The relationship between facts and values
            \item The nature of moral knowledge
            \item The foundations of ethical reasoning
        \end{itemize}
    \end{itemize}
\end{frame}

\begin{frame}{Strengths of Natural Law Theory}
    \begin{itemize}
        \item Natural law theory provides compelling accounts of:
        
        \item Moral Knowledge:
        \begin{itemize}
            \item How we can know moral truth
            \item Why moral knowledge is widely shared
            \item How moral reasoning develops
        \end{itemize}
        
        \item Practical Reasoning:
        \begin{itemize}
            \item The connection between reason and goodness
            \item How to evaluate competing claims
            \item The role of wisdom in decision-making
        \end{itemize}
        
        \item The theory successfully:
            \begin{itemize}
                \item Grounds human rights
                \item Explains moral universals
                \item Bridges theory and practice
            \end{itemize}
    \end{itemize}
\end{frame}

\begin{frame}{Challenges to Natural Law Theory}
    \begin{itemize}
        \item \textbf{Empirical challenges} question natural law's claims about:
        
        \item Human Nature:
        \begin{itemize}
            \item Evolutionary accounts of morality
            \item Cultural diversity in moral beliefs
            \item The role of emotion in moral judgment
        \end{itemize}
        
        \item \textbf{Philosophical challenges} include:
            \begin{itemize}
                \item The fact-value distinction
                \item Moral relativism
                \item Competing accounts of practical reason
            \end{itemize}
    \end{itemize}
    
    \begin{alertblock}{Contemporary Context}
        These challenges have led to sophisticated developments in natural law theory rather than abandonment of the approach.
    \end{alertblock}
\end{frame}

\begin{frame}{Future Directions in Natural Law Theory}
    \begin{itemize}
        \item Contemporary natural law theorists are developing new approaches to:
        
        \item Foundational Questions:
        \begin{itemize}
            \item The nature of practical reason
            \item The relationship between law and morality
            \item The grounds of human rights
        \end{itemize}
        
        \item Emerging Challenges:
        \begin{itemize}
            \item Artificial intelligence and moral agency
            \item Environmental ethics
            \item Global justice
        \end{itemize}
    \end{itemize}
    
    \begin{alertblock}{Enduring Relevance}
        Natural law theory continues to provide valuable insights for addressing contemporary moral and legal challenges while evolving to meet new intellectual demands.
    \end{alertblock}
\end{frame}

\end{document}